%% tikz-nfold.sty
%% Copyright 2023 Jonathan Schulz
%
% This work may be distributed and/or modified under the
% conditions of the LaTeX Project Public License, either version 1.3c
% of this license or (at your option) any later version.
% The latest version of this license is in
% http://www.latex-project.org/lppl.txt
% and version 1.3c or later is part of all distributions of LaTeX
% version 2008-05-04 or later.
%
% This work has the LPPL maintenance status 'maintained'.
%
% The Current Maintainer of this work is Jonathan Schulz.
%
% This work consists of the files pgflibrarybezieroffset.code.tex,
% tikzlibrarynfold.code.tex, tikz-nfold-doc.tex, and tikz-nfold-doc.pdf.

\usetikzlibrary{decorations}
\usetikzlibrary{arrows.meta}
%\usepgflibrary{bezieroffset}
\usepgflibrary{bezieroffset}

%
%
% Intercepting join settings
% --------------------------
%
% The current settings of the line joins are not stored in any TeX registers; instead, direct system
% calls are made to apply the settings. Therefore, we need to modify the pgf macros in order
% to cache the current settings.
%

% According to the pgf documentation, miter limit=10 is the default value,
% but I couldn't find and verify this setting in the pgf code
\gdef\pgf@nfold@cached@miterlimit{10}
\let\pgf@nfold@old@miterlimit\pgfsetmiterlimit
\def\pgfsetmiterlimit#1{%
  \pgf@nfold@old@miterlimit{#1}%
  \edef\pgf@nfold@cached@miterlimit{#1}%
}

% default line join is "miter"
\global\let\pgf@cached@linejoin=m% % b = bevel, m=miter, r=round
\let\pgf@nfold@old@setbeveljoin\pgfsetbeveljoin
\let\pgf@nfold@old@setmiterjoin\pgfsetmiterjoin
\let\pgf@nfold@old@setroundjoin\pgfsetroundjoin
\def\pgfsetbeveljoin{%
  \pgf@nfold@old@setbeveljoin%
  % do NOT change this globally! Needs to be changed back at the end of groups for scoping reasons
  \let\pgf@cached@linejoin=b%
}
\def\pgfsetmiterjoin{%
  \pgf@nfold@old@setmiterjoin%
  \let\pgf@cached@linejoin=m%
}
\def\pgfsetroundjoin{%
  \pgf@nfold@old@setroundjoin%
  \let\pgf@cached@linejoin=r%
}

%
% Various helper commands
% -----------------------

% check if a decoration segment is visible, i.e. not moveto or last
\newif\ifpgf@nfold@segm@visible
\def\ifpgfsegmentvisible#1{%
  \pgf@nfold@segm@visiblefalse%
  \ifx#1\pgfdecorationinputsegmentlineto%
  \pgf@nfold@segm@visibletrue\fi%
  \ifx#1\pgfdecorationinputsegmentcurveto%
  \pgf@nfold@segm@visibletrue\fi%
  \ifx#1\pgfdecorationinputsegmentclosepath%
  \pgf@nfold@segm@visibletrue\fi%
  \ifpgf@nfold@segm@visible%
}


% Joining offset lines
% --------------------
%
% One of the more difficult aspects is joining the segments of an offset path. Without this step,
% the path would be interrupted or self-intersect whenever there is a non-zero angle between
% two segments. This code reproduces the existing line joins "bevel", "miter" and "round" as well
% as possible.
%

\def\pgf@nfold@miterjoin{
  % - Although it isn't obvious, this code is invariant under
  %   \pgf@nfold@deltaphi -> \pgf@nfold@deltaphi + 2pi
  % - The tip of the miter join is computed starting from the unshifted point and then moving 
  \pgfpointadd%
    {\pgf@decorate@inputsegment@first}%
    {\pgfpointpolar%
      {\pgf@nfold@cached@endangle+.5*\pgf@nfold@deltaphi+90}%
      {\pgf@shiftdec@amount/cos(.5*\pgf@nfold@deltaphi)}%
    }%
  \pgfpathlineto{}%
}

\def\pgf@nfold@beveljoin{
  % The subsequent formula needs deltaphi to be between -180 and 180
  \pgfmathsetmacro{\pgf@nfold@deltaphi}{Mod(\pgf@nfold@deltaphi + 180, 360) - 180}
  % For a good-looking result, inside lines should have a constant distance from each other all the way
  % through the bevel join, which is a non-trivial condition. To achieve this result, the outer lines get
  % a bevel-like join and the inner lines get a miter-like join; the threshold depends on the angle.
  % We first compute by how much the outermost line must be continued from the beginning of the join.
  % The protrusion amount must be lowered by a little bit for a rather complicated reason: The offset would be
  % dead on if the outermost offset line were centered on the _edge_ of the wide line, but we want to draw 
  % the outside line _fully inside_ the wide line. The factor of tan(deltaphi/4) can be derived, but is not obvious.
  \pgfmathsetmacro{\bevelouterprotrusion}{\pgf@nfold@shortenstartjoin - .5*\pgflinewidth*abs(tan(.25*\pgf@nfold@deltaphi))}
  % The following applies to middle lines only: We compute by how much they need to be shortened so the distance
  % between the lines in the join is correct.
  \pgfmathsetmacro{\bevelshorten}{2*\insidepercentage*abs(tan(.25*\pgf@nfold@deltaphi))}
  % This theshold decides if the inside line is a bevel or a miter
  \pgfmathparse{\bevelshorten < abs(sin(.5*\pgf@nfold@deltaphi))}
  \ifnum\pgfmathresult=1\relax
    \pgfmathsetmacro{\bevelextension}{\bevelouterprotrusion-\bevelshorten*\pgf@decoration@nfold@hwidth}
    \pgfpointadd{\pgf@nfold@join@start}{\pgfpointpolar{\pgf@nfold@cached@endangle}{\bevelextension}}
    \pgfpathlineto{}
    \pgfpointadd{\pgf@nfold@join@end}{\pgfpointpolar{\pgfdecoratedinputsegmentstartangle}{-\bevelextension}}
    \pgfpathlineto{}
  \else
    \pgf@nfold@miterjoin
  \fi
}

\def\pgf@nfold@roundjoin{
  % The outer half of the lines get arcs, the others get miters
  \ifdim\insidepercentage pt<.5pt\relax
    \pgfpointadd{\pgf@nfold@join@start}{\pgfpointpolar{\pgf@nfold@cached@endangle}{\pgf@nfold@shortenstartjoin}}
    \pgfpathlineto{}
    % Check if the angles are in the correct range; under some conditions we must add or subtract 360
    \pgfmathparse{\turnindicator*(\pgfdecoratedinputsegmentstartangle-\pgf@nfold@cached@endangle)}
    \ifdim\pgfmathresult pt>0pt\relax
      \pgfmathsetmacro{\targetang}{\pgfdecoratedinputsegmentstartangle-\turnindicator*360}
    \else
      \let\targetang\pgfdecoratedinputsegmentstartangle
    \fi
    \pgfpatharc%
      {\pgf@nfold@cached@endangle+90*\turnindicator}%
      {\targetang+90*\turnindicator}%
      {abs(\pgf@shiftdec@fraction)*\pgf@decoration@nfold@hwidth}%
  \else
    \pgf@nfold@miterjoin
  \fi
}

\def\pgf@nfold@make@join{
  % The code must be invariant under deltaphi -> deltaphi + 360, which can be verified experimentally
  \pgfmathsetmacro{\pgf@nfold@deltaphi}{\pgfdecoratedinputsegmentstartangle-\pgf@nfold@cached@endangle}
  % Offset the start and end of this join
  \pgfextract@process\pgf@nfold@join@start{%
    \pgfpointadd{\pgf@nfold@cached@endpoint}
                {\pgfpointpolar{\pgf@nfold@cached@endangle+90}{\pgf@shiftdec@amount}}}%

  \pgfextract@process\pgf@nfold@join@end{%
    \pgfpointadd{\pgf@nfold@segment@start}
                {\pgfpointpolar{\pgfdecoratedinputsegmentstartangle+90}{\pgf@shiftdec@amount}}}%
  \pgfpointdiff{\pgf@nfold@join@start}{\pgf@nfold@join@end}
  % Check if the start of this segment is too close to the end of the previous segment.
  % In that case we don't insert a join segment, as it would look rather glitchy.
  % We use the Manhattan length for performance and to avoid 'dimension too large' errors.
  \pgfmathparse{abs(\pgf@x) + abs(\pgf@y)}
  \ifdim\pgfmathresult pt>0.1pt\relax
    % First step: Check if left or right turn (-1 = left, 1=right)
    \pgfmathsetmacro{\turnindicator}{sign(sin(\pgf@nfold@cached@endangle-\pgfdecoratedinputsegmentstartangle)}
    % between 0 and 1; 0=no distance to cover in the join, 1=maximum distance to cover
    \pgfmathsetmacro{\insidepercentage}{-.5*\turnindicator*\pgf@shiftdec@fraction+.5}
    \if m\pgf@cached@linejoin % because both are only one character, \ifx is not needed
      % miter join
      % First we implement the miter limit: If the angle is too sharp, the miter join is replaced
      % by a bevel join. This is controlled by /tikz/miter limit=..., initially 10.
      \pgfmathsetmacro{\mitercosine}{abs(cos(.5*\pgf@nfold@deltaphi))}
      \pgfmathparse{\mitercosine*\pgf@nfold@cached@miterlimit}
      \ifdim\pgfmathresult pt>1pt\relax
        \pgf@nfold@miterjoin
      \else
        \pgf@nfold@beveljoin
      \fi
    \else
      \if b\pgf@cached@linejoin\relax
        \pgf@nfold@beveljoin
      \else
        \if r\pgf@cached@linejoin
          % round join
          \pgf@nfold@roundjoin
        \fi
      \fi
    \fi
    % All non-trivial joins connect to the end of the join, which is the starting point of the current segment.
    % Note that this entire macro is skipped by if the start and end of the join coincide, so we never create a zero length segment here.
    %
    % There is one edge case here: If two subsequent joins are so close that \pgf@nfold@segment@end
    % and \pgf@nfold@segment@start exchange places *and* we are on the outside of the join 
    % (implying that we are on the inside of the next join), \pgf@nfold@join@end will be located behind the next join.
    % In this case we instead connect to the offset of \pgf@nfold@segment@end (which comes *before* \pgf@nfold@segment@start).
    \pgf@nfold@join@end% default point to connect to
    \ifpgf@nfold@closejoinsedgecase%
      \ifdim\insidepercentage pt<.5pt\relax%
        \pgfpointadd{\pgf@nfold@segment@end}%
        {\pgfpointpolar{\pgfdecoratedinputsegmentendangle+90}{\pgf@shiftdec@amount}}
      \fi%
    \fi
    \pgfpathlineto{}
  \fi
}


%
% Internal offset decoration
% --------------------------
%
% This internal tikz decoration offsets a given path (but doesn't draw it multiple times).
%

% pgfkeys interface, used in the decorations below
\newcount\pgf@nfold@dec@order
\pgf@nfold@dec@order=2
\pgfkeys{
  % - use a counter so we get an error if something other than a number is provided
  % - if needed, throw an error that the intersection library needs to be loaded
  /pgf/decoration/nfold order/.code={
    \pgf@nfold@dec@order=#1
    \ifnum\pgf@nfold@dec@order>\tikz@arrow@intersec@numcached
      \ifdefined\pgfintersectionofpaths\else
        \pgferror{For 'nfold' larger than \tikz@arrow@intersec@numcached\space you need to say \string\usetikzlibrary{intersections}}
      \fi
    \fi
  },
  /pgf/decoration/nfold width/.code={\pgfmathsetlengthmacro{\pgf@decoration@nfold@hwidth}{.5*#1}},
  % integer between 1 and \pgf@nfold@dec@order
  /pgf/decoration/nfold index/.store in=\pgf@nfold@index
}

% In here we cache the end angle of the previous decorated segment, which is otherwise inaccessible
\newdimen\pgf@nfold@cached@endangle
% This stores whether the current segment should begin with a moveto to its offset
\newif\ifpgf@nfold@continuesegment
% This stores whether we are in some edge case of very close joins, see below for details
\newif\ifpgf@nfold@closejoinsedgecase

% some required computations for the current segment
\def\pgf@nfold@shift@prepare@segment{%
  \pgftransformreset % important! otherwise we work in the local coordinate system
  % compute the relative deviation from the original path (between -1.0 and 1.0)
  \pgfmathsetmacro{\pgf@shiftdec@fraction}%
    {-1+2*(\pgf@nfold@index-1)/(\pgf@nfold@dec@order-1)}
  \pgfmathsetlengthmacro{\pgf@shiftdec@amount}{\pgf@decoration@nfold@hwidth*\pgf@shiftdec@fraction}
  % Set some default values; they might get overwritten later
  \let\pgf@nfold@segment@start\pgf@decorate@inputsegment@first
  \let\pgf@nfold@segment@end\pgf@decorate@inputsegment@last
  % Set a default value for \ifpgf@nfold@continuesegment.
  % It may be overwritten by \pgf@nfold@extendtotip
  \ifx\pgfdecorationpreviousinputsegment\pgfdecorationinputsegmentmoveto
    \pgf@nfold@continuesegmentfalse
  \else
    \pgf@nfold@continuesegmenttrue
  \fi
}

% draw the current segment including a join at the start (if present)
\def\pgf@nfold@shift@handle@segment{%
  % In order to make space for the join, it may be necessary to shorten the current segment
  % at the start and/or the end. In here we store by how much the segment needs to be shortened.
  \def\pgf@nfold@shortenstartjoin{0pt}
  \def\pgf@nfold@shortenendjoin{0pt}
  \pgf@nfold@closejoinsedgecasefalse
  % Make a join only if two adjacent segments are both visible
  \ifpgfsegmentvisible\pgfdecorationcurrentinputsegment
    % Step 1.1: Make space for the join at the start if needed
    \ifx\pgfdecorationpreviousinputsegment\pgfdecorationinputsegmentmoveto\else
      \ifdefined\pgfdecorationpreviousinputsegment
        % As far as I am aware, the previous input segment is either moveto or undefined
        \pgfwarning{'nfold': Unexpected previous input segment in decoration 'nfold' (\meaning\pgfdecorationpreviousinputsegment)}
      \fi
      \pgf@xa=\pgfdecoratedinputsegmentstartangle pt\relax
      \advance\pgf@xa by-\pgf@nfold@cached@endangle\relax
      \pgfmathparse{abs(cos(0.5*\pgf@xa))}
      \ifdim\pgfmathresult pt<0.02pt\relax
        % we go full backwards, don't relocate the start
        \pgfwarning{Angle too sharp in decoration 'nfold', expect visual errors}
      \else
        \pgfmathsetlengthmacro{\pgf@nfold@shortenstartjoin}{\pgf@decoration@nfold@hwidth*abs(tan(0.5*\pgf@xa))}
        \pgfextract@process\pgf@nfold@segment@start{%
          \pgfpointadd{\pgf@decorate@inputsegment@first}
                      {\pgfpointpolar{\pgfdecoratedinputsegmentstartangle}{\pgf@nfold@shortenstartjoin}}}%
      \fi
    \fi
    % Step 1.2: Make space for the join at the end if needed
    \ifpgfsegmentvisible\pgfdecorationnextinputsegmentobject
      % for reasons unknown to me,
      % \pgfdecoratedangletonextinputsegment has a "pt" but the others do not
      \pgf@xa=\pgfdecoratedangle pt\relax
      \advance\pgf@xa by \pgfdecoratedangletonextinputsegment\relax
      \advance\pgf@xa by -\pgfdecoratedinputsegmentendangle pt\relax
      \pgfmathparse{abs(cos(0.5*\pgf@xa))}
      \ifdim\pgfmathresult pt<0.02pt\relax
        \pgfwarning{Angle too sharp in decoration 'nfold', expect visual errors}
      \else
        \pgfmathsetlengthmacro{\pgf@nfold@shortenendjoin}{\pgf@decoration@nfold@hwidth*abs(tan(0.5*\pgf@xa))}
        \pgfextract@process\pgf@nfold@segment@end{%
          \pgfpointadd{\pgf@decorate@inputsegment@last}
                      {\pgfpointpolar{\pgfdecoratedinputsegmentendangle}{-\pgf@nfold@shortenendjoin}}}%
      \fi
      % Step 1.3: Detect an edge case
      % This edge case appears whenever the current segment is such a short line that we would have to reduce its length
      % to less than zero to make space for the joins. In such cases, the line is not drawn at all, and slight modifications
      % must be made to the joins for a correct output (i.e. one join is immediately followed by the next without a segment in between).
      %
      % This edge case can appear for curves as well, but they are much harder to deal with.
      \ifx\pgfdecorationcurrentinputsegment\pgfdecorationinputsegmentlineto
        % No need to check for \pgfdecorationinputsegmentclosepath, as it should not be followed by any further segments
        % Now: Check if the total amount of shortening is larger than the length of the segment
        \pgfpointdiff{\pgf@decorate@inputsegment@first}{\pgf@decorate@inputsegment@last}
        \pgfmathparse{\pgf@nfold@shortenstartjoin+\pgf@nfold@shortenendjoin > veclen(\pgf@x,\pgf@y)}
        \ifnum\pgfmathresult=1\relax
          \pgf@nfold@closejoinsedgecasetrue
        \fi
      \fi
    \fi% end if next segment visible
    %
    % Step 2: Draw the join at the start if applicable
    %
    \ifx\pgfdecorationpreviousinputsegment\pgfdecorationinputsegmentmoveto
      \ifpgf@nfold@closejoinsedgecase
        % If the previous segment is a moveto and the current segment is a "close joins" edge case,
        % nothing needs to be drawn here (the relevant draw call will be made at the join of the subsequent
        % segment). We must therefore make sure that we move to the correct end point of this segment.
        % Counterintuitively, this is given by the offset of \pgf@nfold@segment@start since the start and end
        % are reversed in the edge case.
        \pgfpointadd%
          {\pgf@nfold@segment@start}%
          {\pgfpointpolar{\pgfdecoratedinputsegmentendangle+90}{\pgf@shiftdec@amount}}
        \pgfpathmoveto{}
       \fi
    \else
      \pgf@nfold@make@join
    \fi
  \fi% end if current segment visible
  %
  % Step 3: Draw the new segment.
  %
  % The value of \ifpgf@nfold@continuesegment decides whether we start with a moveto.
  \ifpgf@nfold@closejoinsedgecase\else
    \ifx\pgfdecorationcurrentinputsegment\pgfdecorationinputsegmentlineto
      \ifpgf@nfold@continuesegment
        \pgfoffsetlinenomove{\pgf@nfold@segment@start}{\pgf@nfold@segment@end}{\pgf@shiftdec@amount}
      \else
        \pgfoffsetline{\pgf@nfold@segment@start}{\pgf@nfold@segment@end}{\pgf@shiftdec@amount}
      \fi
    \fi
    \ifx\pgfdecorationcurrentinputsegment\pgfdecorationinputsegmentclosepath
      \pgfwarning{'nfold': The option 'cycle' is not yet properly supported.}
      \ifpgf@nfold@continuesegment
        \pgfoffsetlinenomove{\pgf@nfold@segment@start}{\pgf@nfold@segment@end}{\pgf@shiftdec@amount}
      \else
        \pgfoffsetline{\pgf@nfold@segment@start}{\pgf@nfold@segment@end}{\pgf@shiftdec@amount}
      \fi
    \fi
    \ifx\pgfdecorationcurrentinputsegment\pgfdecorationinputsegmentcurveto
      \ifpgf@nfold@continuesegment
        \pgfoffsetcurvenomove{\pgf@nfold@segment@start}{\pgf@decorate@inputsegment@supporta}{\pgf@decorate@inputsegment@supportb}{\pgf@nfold@segment@end}{\pgf@shiftdec@amount}
      \else
        \pgfoffsetcurve{\pgf@nfold@segment@start}{\pgf@decorate@inputsegment@supporta}{\pgf@decorate@inputsegment@supportb}{\pgf@nfold@segment@end}{\pgf@shiftdec@amount}
      \fi
    \fi
    \ifx\pgfdecorationcurrentinputsegment\pgfdecorationinputsegmentlast
      % Seems like this never happens
      \pgfwarning{'nfold': Unexpected state (last part in segment state).}
    \fi
  \fi% end if edge case
  % cache the end angle and end point because we need it on the next pass
  \global\pgf@nfold@cached@endangle=\pgfdecoratedinputsegmentendangle pt\relax
  \ifx\pgfdecorationnextinputsegmentobject\pgfdecorationinputsegmentmoveto\else
    \global\let\pgf@nfold@cached@endpoint\pgf@nfold@segment@end
  \fi
  % Step 4: Extend into the arrow tip at the end (if present)
  \ifx\pgfdecorationnextinputsegmentobject\pgfdecorationinputsegmentlast%
    \pgf@nfold@check@tip@end%
    \ifpgf@nfold@tip@implies%
      \pgf@nfold@extendtotip{e}%
    \fi%
  \fi%
}

\pgfdeclaredecoration{pgf@nfold@shift}{start}{%
  \state{start}[width=\pgfdecoratedinputsegmentremainingdistance, next state=segment]{%
    \egroup\begingroup
    \pgf@nfold@shift@prepare@segment
    \pgf@nfold@check@tip@start%
    \ifpgf@nfold@tip@implies%
      \pgf@nfold@extendtotip{s}%
    \fi
    \pgf@nfold@shift@handle@segment
    \endgroup\bgroup%
  }
  \state{segment}[width=\pgfdecoratedinputsegmentremainingdistance, next state=segment]{%
    \egroup\begingroup
    \pgf@nfold@shift@prepare@segment
    \pgf@nfold@shift@handle@segment
    \endgroup\bgroup%
  }%
}%



% Internal pre-pass decoration
% ----------------------------
%
% This internal decoration is run as the first step in the decoration 'nfold'.
% It caches some data that won't be accessible later, and it also makes space for the arrow tips
% if needed.
%

\def\pgf@nfold@storedata@handle@segment{%
  \let\pgf@nfold@segment@end\pgf@decorate@inputsegment@last%
  \ifx\pgfdecorationnextinputsegmentobject\pgfdecorationinputsegmentlast%
    % Implementing shorten > and making space for the arrow tip (if present)
    \pgf@nfold@check@tip@end%
    \ifpgf@nfold@tip@implies%
      \pgfmathparse{-\pgf@shorten@end@additional-2.06*\pgf@decoration@nfold@hwidth-0.5*\pgflinewidth}
    \else
      \pgfmathparse{-\pgf@shorten@end@additional}
    \fi
    % This is the point where the arrow body ends
    \pgfextract@process\pgf@nfold@segment@end{%
      \pgfpointadd{\pgf@decorate@inputsegment@last}%
                  {\pgfpointpolar{\pgfdecoratedinputsegmentendangle}{\pgfmathresult pt}}}%
    % This is the tip of the arrow, required for drawing the arrow head
    \pgfextract@process\pgf@nfold@original@last{%
      \pgfpointadd%
        {\pgf@decorate@inputsegment@last}%
        {\pgfpointpolar%
          {\pgfdecoratedinputsegmentendangle}%
          {-\pgf@shorten@end@additional}}}%
    \global\let\pgf@nfold@original@last\pgf@nfold@original@last
  \fi%
  \ifx\pgfdecorationcurrentinputsegment\pgfdecorationinputsegmentmoveto%
    \pgfpathmoveto{\pgf@nfold@segment@end}%
  \fi%
  \ifx\pgfdecorationcurrentinputsegment\pgfdecorationinputsegmentlineto%
    \pgfpathlineto{\pgf@nfold@segment@end}%
  \fi%
  \ifx\pgfdecorationcurrentinputsegment\pgfdecorationinputsegmentclosepath%
    \pgfpathlineto{\pgf@nfold@segment@end}%
  \fi%
  \ifx\pgfdecorationcurrentinputsegment\pgfdecorationinputsegmentcurveto%
    \pgfpathcurveto%
      {\pgf@decorate@inputsegment@supporta}%
      {\pgf@decorate@inputsegment@supportb}%
      {\pgf@nfold@segment@end}%
  \fi%
}

\pgfdeclaredecoration{pgf@nfold@storedata}{start}{%
  \state{start}[width=\pgfdecoratedinputsegmentremainingdistance, next state=segment]{%
    \egroup\begingroup
    \pgftransformreset
    \iftikz@mode@double\else
      \pgferror{Must enable /tikz/double to use 'nfold' (e.g. using [double distance=5pt]).}
    \fi
    % Cache arrow settings
    \pgf@nfold@storearrows
    % Cache inner and outer line width from the '/tikz/double' settings. These are stored in
    % the macro \tikz@double@setup. Internally, the value of \pgfinnerlinewidth determines whether 
    % double stroke is enabled or not. Calling \tikz@double@setup changes line widths globally, so
    % encapsulating the calls in a group would still have side effects. What we do instead is call
    % \tikz@double@setup and then restore \pgflinewidth.
    \tikz@double@setup
    \pgfmathsetlengthmacro{\pgf@nfold@temp@fullwidth}{.25*(\pgflinewidth+\pgfinnerlinewidth)}
    \global\edef\pgf@decoration@nfold@hwidth{\pgf@nfold@temp@fullwidth}
    % We can either store \pgflinewidth before calling \tikz@double@setup, or we can reconstruct
    % its old value from the new values. We do the latter here.
    \pgfmathparse{.5*(\pgflinewidth-\pgfinnerlinewidth)}
    \pgfsetlinewidth{\pgfmathresult pt}
    % shorten < and shorten > do not work well with this decoration for various reasons.
    % We "bake" them into the path in the pre-pass and then disable them for the rendering passes.
    \pgfextract@process\pgf@nfold@original@first{%
      \pgfpointadd%
        {\pgf@decorate@inputsegment@first}%
        {\pgfpointpolar{\pgfdecoratedinputsegmentstartangle}{\pgf@shorten@start@additional}}}%
    % store the tip of the arrow
    \global\let\pgf@nfold@original@first\pgf@nfold@original@first%
    % make space if applicable
    \pgf@nfold@check@tip@start%
    \ifpgf@nfold@tip@implies%
      \pgfmathparse{\pgf@shorten@start@additional+2.06*\pgf@decoration@nfold@hwidth+0.5*\pgflinewidth}
    \else
      \pgfmathparse{\pgf@shorten@start@additional}
    \fi
    \pgfpointadd%
      {\pgf@decorate@inputsegment@first}%
      {\pgfpointpolar{\pgfdecoratedinputsegmentstartangle}{\pgfmathresult pt}}
    %
    \pgfpathmoveto{}
    \pgf@nfold@storedata@handle@segment
    \endgroup\bgroup
  }
  \state{segment}[width=\pgfdecoratedinputsegmentremainingdistance, next state=segment]{%
    \egroup\begingroup
    \pgftransformreset
    \pgf@nfold@storedata@handle@segment
    \endgroup\bgroup
  }%
  \state{final}{}
}%


% Detecting arrow tips
% --------------------

% Old arrow matching
%
% This is the content of \pgf@arrow@tip@sequence for arrows | and Implies, respectively
%
%\edef\pgf@arrow@macro@verticalline{\noexpand\pgf@arrow@handle@shorthand@empty {\expandafter\noexpand\csname pgf@ar@means@|\endcsname }}
%\def\pgf@arrow@macro@Implies{\pgf@arrow@handle {Implies}{}}
% The next one is wrong; use \csname pgf@ar@means@tikzcd implies\endcsname
%\def\tikzcd@arrow@implies{\pgf@arrow@handle@shorthand@empty {\pgf@ar@means@tikzcd implies }}

% Intercept the arrows at definition
\let\pgf@nfold@cachedarrows\pgfutil@empty % default value
\let\pgf@nfold@oldsetarrows\pgfsetarrows
\def\pgfsetarrows#1{%
  \def\pgf@nfold@cachedarrows{#1}%
  \pgf@nfold@oldsetarrows{#1}%
}

\let\pgf@nfold@cached@arrow@start\pgfutil@empty
\let\pgf@nfold@cached@arrow@end\pgfutil@empty
\def\pgf@nfold@storearrows{%
  \ifx\pgf@nfold@cachedarrows\pgfutil@empty
    % reset the cached values in case we don't have an arrow; this is important
    % because the cached values are set globally
    \global\let\pgf@nfold@cached@arrow@start\pgfutil@empty
    \global\let\pgf@nfold@cached@arrow@end\pgfutil@empty
  \else%
    \expandafter\pgf@nfold@parsearrows@\pgf@nfold@cachedarrows\pgf@stop%
  \fi%
}
\def\pgf@nfold@parsearrows@#1-#2\pgf@stop{%
  % These must be set globally because \pgf@nfold@storearrows is called from within a decoration;
  % also, this cached value must survive subsequent calls of \pgfsetarrows{} which disable arrows
  \gdef\pgf@nfold@cached@arrow@start{#1}%
  \gdef\pgf@nfold@cached@arrow@end{#2}%
}

% Here we match against the names of the arrows that were set in \pgfarrowsset{...}.
% This does not respect aliases, i.e. redefining /tikzcd implies cap will not have
% the desired effect. We may be able to fix this in the future, see below.
\def\pgf@nfold@nonetip@i{tikzcd implies cap}
\def\pgf@nfold@impliestip@i{Implies}
\def\pgf@nfold@impliestip@ii{tikzcd implies}
\def\pgf@nfold@mapstotip@i{|}
\def\pgf@nfold@mapstotip@ii{tikzcd implies bar}
\def\pgf@nfold@mapstotip@iii{Bar}
\newif\ifpgf@nfold@tip@implies
\newif\ifpgf@nfold@tip@mapsto

\def\pgf@nfold@check@tip@start{%
\pgf@nfold@check@tip{\pgf@nfold@cached@arrow@start}%
}
\def\pgf@nfold@check@tip@end{%
\pgf@nfold@check@tip{\pgf@nfold@cached@arrow@end}%
}

\def\pgf@nfold@check@tip#1{%
  \pgf@nfold@tip@impliesfalse%
  \pgf@nfold@tip@mapstofalse%
  \ifx#1\pgfutil@empty%
    % no tip
  \else
    \ifx#1\pgf@nfold@nonetip@i%
      % no tip
    \else
      \ifx#1\pgf@nfold@impliestip@i%
        \pgf@nfold@tip@impliestrue%
      \else
        \ifx#1\pgf@nfold@impliestip@ii%
          \pgf@nfold@tip@impliestrue%
        \else
          \ifx#1\pgf@nfold@mapstotip@i%
            \pgf@nfold@tip@mapstotrue%
          \else
            \ifx#1\pgf@nfold@mapstotip@ii%
              \pgf@nfold@tip@mapstotrue%
            \else
              \ifx#1\pgf@nfold@mapstotip@iii%
                \pgf@nfold@tip@mapstotrue%
               \else
                 \pgfwarning{'nfold': Unsupported arrow tip "#1"}
              \fi
            \fi
          \fi
        \fi
      \fi
    \fi
  \fi
}

% Future: In principle, we can dereference aliases (means=...) the following way:
%
%    \ifcsname pgf@ar@means@tikzcd implies bar\endcsname
%      \expandafter\let\expandafter\tempmacro\csname pgf@ar@means@tikzcd implies bar\endcsname
%      \pgfwarning{\meaning\tempmacro}
%    \fi
%
% However, the alias is not a name, but a macro, which in turn could call dereference aliases.
% We could go down this rabbit hole in the future, but I won't do that for now.
%
% For now this means that redefining /tikz/commutative diagrams/tikzcd implies cap will not
% have the desired effect.


% Rendering arrow tips
% --------------------

% Precomputed intersections
%
% For arrows of order n > 2 with an Implies tip, the constituent parts of the n-fold arrow
% end somewhere in the middle of the tip. The exact end point must be computed using
% the intersections library. To speed up compilation times, the intersection points are precomputed
% up to n = 5. If your document contains arrows of order 6 or larger, consider adding those
% as well; the values are output in the log file.
\expandafter\def\csname tikz@arrow@intersec@cache@2@3\endcsname{\pgfqpoint{2pt}{0pt}}
\expandafter\def\csname tikz@arrow@intersec@cache@2@4\endcsname{\pgfqpoint{0.94063pt}{-0.33333pt}}
\expandafter\def\csname tikz@arrow@intersec@cache@3@4\endcsname{\pgfqpoint{0.94063pt}{0.33333pt}}
\expandafter\def\csname tikz@arrow@intersec@cache@2@5\endcsname{\pgfqpoint{0.64167pt}{-0.5pt}}
\expandafter\def\csname tikz@arrow@intersec@cache@3@5\endcsname{\pgfqpoint{2pt}{0pt}}
\expandafter\def\csname tikz@arrow@intersec@cache@4@5\endcsname{\pgfqpoint{0.64167pt}{0.5pt}}
% intersections are precomputed up to this order
\def\tikz@arrow@intersec@numcached{5}

% Decoration for the start tip
\pgfdeclaredecoration{pgf@nfold@tip@start}{start}{%
  \state{start}[width=\pgfdecoratedinputsegmentremainingdistance, next state=segment]{%
    \egroup\begingroup%
    \pgf@nfold@check@tip@start%
    \pgf@nfold@setup@start@tip%
    \ifpgf@nfold@tip@implies%
      \pgf@nfold@setupimplies%
      \pgf@nfold@drawimplies%
    \fi%
    \ifpgf@nfold@tip@mapsto%
      \pgf@nfold@setupmapsto%
      \pgf@nfold@drawmapsto%
    \fi%
    \endgroup\bgroup%
  }%
  \state{segment}[width=\pgfdecoratedinputsegmentremainingdistance, next state=segment]{}%
}%

% Decoration for the end tip
\pgfdeclaredecoration{pgf@nfold@tip@end}{segment}{%
  \state{segment}[width=\pgfdecoratedinputsegmentremainingdistance, next state=segment]{}
  \state{final}{%
    \egroup\begingroup%
    \pgf@nfold@check@tip@end% this step already checks for unsupported arrows
    \pgf@nfold@setup@end@tip
    \ifpgf@nfold@tip@implies%
      \pgf@nfold@setupimplies%
      \pgf@nfold@drawimplies%
    \fi%
    \ifpgf@nfold@tip@mapsto
      \pgf@nfold@setupmapsto
      \pgf@nfold@drawmapsto
    \fi
    \endgroup\bgroup%
  }
}%


\def\pgf@nfold@setup@start@tip{
  \pgftransformreset
  \pgftransformshift{\pgf@nfold@original@first}
  \pgftransformrotate{\pgfdecoratedinputsegmentstartangle}
  \pgftransformshift{\pgfpoint{.5*\pgflinewidth}{0pt}}
  \pgftransformscale{\pgf@decoration@nfold@hwidth}
  \pgftransformxscale{-1.}
}
\def\pgf@nfold@setup@end@tip{
  \pgftransformreset
  \pgftransformshift{\pgf@nfold@original@last}
  \pgftransformrotate{\pgf@nfold@cached@endangle}
  \pgftransformshift{\pgfpoint{-.5*\pgflinewidth}{0pt}}
  \pgftransformscale{\pgf@decoration@nfold@hwidth}
}

\def\pgf@nfold@setupimplies{%
  \pgfsetroundjoin%
  \pgfsetroundcap%
}
\def\pgf@nfold@drawimplies{
  \pgftransformshift{\pgfqpoint{-2pt}{0pt}}
  \pgfpathmoveto{\pgfqpoint{-1.4pt}{2.65pt}}
  \pgfpathcurveto{\pgfqpoint{-0.75pt}{1.25pt}}{\pgfqpoint{1pt}{0.05pt}}{\pgfqpoint{2pt}{0pt}}
  \pgfpathcurveto{\pgfqpoint{1pt}{-0.05pt}}{\pgfqpoint{-0.75pt}{-1.25pt}}{\pgfqpoint{-1.4pt}{-2.65pt}}
}
\def\pgf@nfold@setupmapsto{%
  \pgfsetbuttcap%
}
\def\pgf@nfold@drawmapsto{
  % Try: use the same width as the arrow head, minus the caps
  \pgfpathmoveto{\pgfqpoint{0pt}{2.65pt}}
  \pgfpathlineto{\pgfqpoint{0pt}{-2.65pt}}
}

% This macro extends the arrow body to the tips
% parameter: s=start, e=end
\def\pgf@nfold@extendtotip#1{
  % Do not extend the arrow for index=1 and index=order, it already ends in the right place
  \ifnum\pgf@nfold@index>1\ifnum\pgf@nfold@index<\pgf@nfold@dec@order\relax%
    % Step 1: Find the intersection of the arrow's path with the head. This is is computationally
    % expensive, so we first look if the value has been precomputed.
    \ifcsname tikz@arrow@intersec@cache@\pgf@nfold@index @\the\pgf@nfold@dec@order\endcsname
      \pgfextract@process\pgf@nfold@arrowintersect
        {\csname tikz@arrow@intersec@cache@\pgf@nfold@index @\the\pgf@nfold@dec@order\endcsname}%
    \else
      % the intersection has not been precomputed, thus compute on the fly here
      \pgfintersectionofpaths{
        % specify the tip
        \pgf@nfold@drawimplies
      }{
        % extend the body to intersect the tip
        \pgfpathmoveto{\pgfqpoint{-3pt}{\pgf@shiftdec@fraction pt}}
        \pgfpathlineto{\pgfqpoint{3pt}{\pgf@shiftdec@fraction pt}}
      }
      \ifnum\pgfintersectionsolutions>0
        \pgfextract@process\pgf@nfold@arrowintersect{\pgfpointintersectionsolution{1}}%
        \makeatother
        \typeout{tikz-nfold: computed intersection cache@\pgf@nfold@index @\the\pgf@nfold@dec@order: \string\pgfqpoint{\the\pgf@x}{\the\pgf@y}^^J}
        \makeatletter
      \else
        % this is a failsafe and should never be reached
        \pgfwarning{'nfold': did not find intersection}
        \pgfextract@process\pgf@nfold@arrowintersect{\pgfqpoint{0pt}{\pgf@shiftdec@fraction pt}}%
      \fi
    \fi% if precomputed
  % Step 2: Extend the arrow body to the intersection point.
  % If the tip is at the beginning of the path, we have to move to the intersection
  % and then draw a line to the "regular" starting point. The subsequent segment then
  % should omit its moveto.
  % If the tip is at the end, we are already in the right position and only need to extend
  % the current path to the intersection point.
  \pgftransformreset
  \if#1s
    \pgftransformshift{\pgf@nfold@original@first}
    \pgftransformrotate{\pgfdecoratedinputsegmentstartangle}
    \pgftransformxscale{-1}
  \else
    \pgftransformshift{\pgf@nfold@original@last}
    \pgftransformrotate{\pgfdecoratedinputsegmentendangle}
  \fi
  \pgftransformshift{\pgfpoint{-2*\pgf@decoration@nfold@hwidth-.5*\pgflinewidth}{0pt}}
  \pgftransformscale{\pgf@decoration@nfold@hwidth}
  \if#1s
    \pgfpathmoveto{\pgf@nfold@arrowintersect}
    % This is precisely the start of the body, shifted vertically
    \pgfpathlineto{\pgfqpoint{-0.06pt}{\pgf@shiftdec@fraction pt}}
    \pgf@nfold@continuesegmenttrue
  \else\if#1e
    \pgfpathlineto{\pgf@nfold@arrowintersect}
  \else
    \pgferror{Invalid argument to \string\pgf@nfold@extendtotip: \meaning#1}
  \fi\fi
  \pgftransformreset
\fi\fi% if 1 < i < nArrows
}


%
% n-fold decoration
% -----------------
%
% This decoration is intended to be used by the end user. It replaces a given path by n parallel versions of the path.

\newcount\tikz@nfold@loop@index
\def\tikz@nfold@run@loop#1{%
  \edef\tikz@nfold@postaction{%
    \tikz@nfold@postaction% this already includes a comma
    postaction={%
      draw,arrows=-,shorten <=0pt,shorten >=0pt,% disable shorten, is implemented in the pre-pass
      decorate,decoration={%
        pgf@nfold@shift, nfold order=#1, nfold index=\the\tikz@nfold@loop@index%
      }%
    },%
  }
  \advance\tikz@nfold@loop@index by -1\relax
  \ifnum\tikz@nfold@loop@index>0%
    \tikz@nfold@run@loop{#1}%
  \fi%
}
\tikzset{
  nfold/.code={
    \tikzset{draw=none}
    \pgf@nfold@dec@order=#1\relax
    \ifnum\pgf@nfold@dec@order<2
      \pgferror{'nfold' must take a value of at least 2, got \the\pgf@nfold@dec@order}
    \fi
    \tikz@nfold@loop@index=\pgf@nfold@dec@order
    % The final style has the following form:
    % [draw=none,
    %  postaction={
    %   draw=none,decorate,decoration=pgf@nfold@storedata,
    %   postaction={draw,arrows=-,decorate,decoration={pgf@nfold@shift,nfold order=#1,nfold index=1}},
    %   [...],
    %   postaction={draw,arrows=-,decorate,decoration={pgf@nfold@shift,nfold order=#1,nfold index=#1}}
    %  }
    % ]
    % We build the postactions piece by piece in the macro \tikz@nfold@postaction. Then we apply it
    % using \tikzset and .expand once, so potential future changes to \tikz@nfold@postaction do not 
    % affect the settings in tikz.
    %
    \def\tikz@nfold@postaction{}
    \tikz@nfold@run@loop{#1}
    \edef\tikz@nfold@postaction{%
        draw=none,decorate,decoration=pgf@nfold@storedata,%
        \tikz@nfold@postaction%
        postaction={
          draw,arrows=-,solid,shorten <=0pt,shorten >=0pt,
          decorate,decoration=pgf@nfold@tip@start},%
        postaction={
          draw,arrows=-,solid,shorten <=0pt,shorten >=0pt,
          decorate,decoration=pgf@nfold@tip@end},%
    }
    \tikzset{postaction/.expand once=\tikz@nfold@postaction}
  },
  nfold/.default=2
}

\endinput
