%% tikz-nfold.sty
%% Copyright 2023 Jonathan Schulz
%
% This work may be distributed and/or modified under the
% conditions of the LaTeX Project Public License, either version 1.3c
% of this license or (at your option) any later version.
% The latest version of this license is in
% http://www.latex-project.org/lppl.txt
% and version 1.3c or later is part of all distributions of LaTeX
% version 2008-05-04 or later.
%
% This work has the LPPL maintenance status 'maintained'.
%
% The Current Maintainer of this work is Jonathan Schulz.
%
% This work consists of the files pgflibrarybezieroffset.code.tex,
% tikzlibrarynfold.code.tex, tikz-nfold-doc.tex, and tikz-nfold-doc.pdf.

\usetikzlibrary{arrows.meta}
\usepgflibrary{bezieroffset}


%
% General idea
% ============
%
% In order to offset a path, we must first analyse and slightly modify it. For example,
% the start and end points of segments need to be relocated slightly in order to make room
% for the joins, and most curves must be subdivided for the offsetting algorithm. To avoid
% redundant computations in n-fold paths, we pre-compute as much data as possible and store
% all this information in a "parsed path" macro. The structure of the parsed path resembles
% the structure of a pgf softpath, but it consists of different tokens.
%



%
% Intercepting join settings
% --------------------------
%
% The current settings of the line joins are not stored in any TeX registers; instead, direct system
% calls are made to apply the settings. Therefore, we need to modify the pgf macros in order
% to cache the current settings.
%

% According to the pgf documentation, miter limit=10 is the default value,
% but I couldn't find and verify this setting in the pgf code
\gdef\pgf@nfold@cached@miterlimit{10}
\let\pgf@nfold@old@miterlimit\pgfsetmiterlimit
\def\pgfsetmiterlimit#1{%
  \pgf@nfold@old@miterlimit{#1}%
  \edef\pgf@nfold@cached@miterlimit{#1}%
}

% default line join is "miter"
\global\let\pgf@cached@linejoin=m% % b = bevel, m=miter, r=round
\let\pgf@nfold@old@setbeveljoin\pgfsetbeveljoin
\let\pgf@nfold@old@setmiterjoin\pgfsetmiterjoin
\let\pgf@nfold@old@setroundjoin\pgfsetroundjoin
\def\pgfsetbeveljoin{%
  \pgf@nfold@old@setbeveljoin%
  % do NOT change this globally! Needs to be changed back at the end of groups for scoping reasons
  \let\pgf@cached@linejoin=b%
}
\def\pgfsetmiterjoin{%
  \pgf@nfold@old@setmiterjoin%
  \let\pgf@cached@linejoin=m%
}
\def\pgfsetroundjoin{%
  \pgf@nfold@old@setroundjoin%
  \let\pgf@cached@linejoin=r%
}


%
% Joining offset lines
% --------------------
%
% One of the more difficult aspects is joining the segments of an offset path. Without this step,
% the path would be interrupted or self-intersect whenever there is a non-zero angle between
% two segments. This code reproduces the existing line joins "bevel", "miter" and "round".
%

\def\pgf@nfold@miterjoin{
  % The tip of the miter join is computed starting from the original (unshifted) centre of the join;
  % we then move orthorgonal to the average of the old and new angle
  \pgfpathlineto{
    \pgfpointadd{\pgf@nfold@join@centre}{%
      \pgfpointpolar{\pgf@nfold@firstang+.5*\pgf@nfold@deltaphi+90}% do not change
        {\pgf@nfold@shiftamount/cos(.5*\pgf@nfold@deltaphi)}%
    }%
  }%
}

\def\pgf@nfold@beveljoin{
  % The bevel join for one component line consists of three parts:
  % 1) an extension of the ingoing line,
  % 2) a middle line, angled at the average of the incoming and outgoing line,
  % 3) an extension of the outgoing line.
  % Different components of the ingoing and outgoing lines have a constant distance from each other.
  % For a good-looking output, the mittle parts of the component lines thus should also have a constant distance,
  % which is a non-trivial condition. To generate such an output the outer lines get a bevel-like join
  % and the inner lines get a miter-like join; the threshold depends on deltaphi.
  % We first compute by how much the outermost line must be continued from the beginning of the join.
  % The protrusion amount must be lowered by a little bit for a rather complicated reason: The offset would be
  % dead on if the outermost offset line were centered on the _edge_ of the wide line, but we want to draw 
  % the outside line _fully inside_ the wide line. The factor of tan(deltaphi/4) can be derived, but is not obvious.
  \pgfmathsetlengthmacro{\bevelouterprotrusion}%
    {\pgf@nfold@shortenstartjoin pt - .5*\pgflinewidth*abs(tan(.25*\pgf@nfold@deltaphi))}
  % The following applies to middle lines only: We compute by how much they need to be shortened so the distance
  % between the lines in the join is correct.
  \pgfmathsetmacro{\bevelshorten}{2*\insidepercentage*abs(tan(.25*\pgf@nfold@deltaphi))}
  % This threshold decides if the inside line has a bevel or a miter join
  \pgfmathparse{\bevelshorten < abs(sin(.5*\pgf@nfold@deltaphi))}
  \ifnum\pgfmathresult=1\relax
    \pgfmathsetlengthmacro{\bevelextension}{\bevelouterprotrusion-\bevelshorten*\pgf@nfold@hwidth}
    \pgfpointadd{\pgf@nfold@join@start}{\pgfpointpolar{\pgf@nfold@firstang}{\bevelextension}}
    \pgfpathlineto{}
    \pgfpointadd{\pgf@nfold@join@end}{\pgfpointpolar{\pgf@nfold@secondang}{-\bevelextension}}
    \pgfpathlineto{}
  \else
    \pgf@nfold@miterjoin
  \fi
}

\def\pgf@nfold@roundjoin{
  % The outer half of the lines get arcs, the others get miters
  \ifdim\insidepercentage pt<.5pt\relax
    \pgfpointadd{\pgf@nfold@join@start}{\pgfpointpolar{\pgf@nfold@firstang}{\pgf@nfold@shortenstartjoin}}
    \pgfpathlineto{}
    % TODO needs unit tests for all cases left, right, across the 360 gap etc.
    \pgfmathsetmacro\pgf@tmp@firstang{\pgf@nfold@firstang+90*\turnindicator}
    \pgfpatharc%
      {\pgf@tmp@firstang}%
      {\pgf@tmp@firstang+\pgf@nfold@deltaphi}%
      {abs(\pgf@nfold@shift@fraction)*\pgf@nfold@hwidth}%
  \else
    \pgf@nfold@miterjoin
  \fi
}

% The handler that will be added to the parsed path
% #1: The centre of the join
% #2: The end point of the first segment
% #3: The starting point of the second segment
% #4: The angle of the first segment
% #5: The angle of the second segment
% #6: deltaphi in the range [-180, 180]
% #7: the distance between #1 and #2
% #8: the join + finish macro
\def\pgf@nfold@token@join#1#2#3#4#5#6#7#8{%
  \def\pgf@nfold@join@centre{#1}%
  \def\pgf@nfold@join@prevend{#2}%
  \def\pgf@nfold@join@nextstart{#3}%
  \def\pgf@nfold@firstang{#4}%
  \def\pgf@nfold@secondang{#5}%
  % Offset the start and end of this join
  \pgfextract@process\pgf@nfold@join@start{%
    \pgfpointadd{#2}{\pgfpointpolar{\pgf@nfold@firstang+90}{\pgf@nfold@shiftamount}}}%
  \pgfextract@process\pgf@nfold@join@end{%
    \pgfpointadd{#3}{\pgfpointpolar{\pgf@nfold@secondang+90}{\pgf@nfold@shiftamount}}}%
  \pgf@process{\pgfpointdiff{\pgf@nfold@join@start}{\pgf@nfold@join@end}}
  % Check if the start of this segment is very close to the end of the previous segment.
  % In that case we don't need a join at all
  \pgfpointtaxicabnorm\pgfutil@tempdima
  \ifdim\pgfutil@tempdima>0.1pt\relax
    \def\pgf@nfold@deltaphi{#6}%
    \def\pgf@nfold@shortenstartjoin{#7}%
    % First step: Check if left or right turn (-1=left, 1=right)
    \ifdim\pgf@nfold@deltaphi pt<0pt
      \def\turnindicator{1}
    \else
      \def\turnindicator{-1}
    \fi
    % \insidepercentage: between 0.0 and 1.0;
    % 0=no distance to cover in the join, 1=maximum distance to cover
    \pgfutil@tempdima=\pgf@nfold@shift@fraction pt\relax
    \pgfutil@tempdima=\turnindicator\pgfutil@tempdima
    \advance\pgfutil@tempdima by-1pt\relax
    \pgfutil@tempdima=-.5\pgfutil@tempdima
    % \insidepercentage  = .5 * (1 - \turnindicator*\pgf@nfold@shift@fraction)
    \edef\insidepercentage{\pgf@sys@tonumber\pgfutil@tempdima}%
    #8%
  \fi
}


%
% Parser: Join handler
%
\def\pgf@nfold@parser@handlejoin{%
  % \pgf@xa := abs(deltaphi@start)
  \pgf@xa=\pgf@nfold@deltaphi@start pt\relax
  \ifdim\pgf@xa<0pt\relax
    \pgf@xa=-\pgf@xa
  \fi
  % Skip the entire join if abs(deltaphi) is too small
  \ifdim\pgf@xa>1pt\relax
    \edef\pgf@nfold@jointype{\pgf@cached@linejoin}
    \if\pgf@cached@linejoin m% \ifx is not needed because both are only one character
      % miter join
      % First we implement the miter limit: If the angle is too sharp, the miter join is replaced
      % by a bevel join. This is controlled by /tikz/miter limit=..., initially 10.
      \pgf@xa=.5\pgf@xa
      \pgfmathcos@{\pgf@sys@tonumber\pgf@xa}
      \pgf@xa=\pgfmathresult pt\relax
      \pgf@xa=\pgf@nfold@cached@miterlimit\pgf@xa
      % Switch to bevel if miterlimit*cos(.5*abs(deltaphi)) <= 1
      \ifdim\pgf@xa>1pt\relax
        \def\pgf@nfold@tmp@joinmacro{\pgf@nfold@miterjoin}
      \else
        \def\pgf@nfold@tmp@joinmacro{\pgf@nfold@beveljoin}
      \fi
    \else
      \if b\pgf@cached@linejoin\relax
        \def\pgf@nfold@tmp@joinmacro{\pgf@nfold@beveljoin}
      \else
        \def\pgf@nfold@tmp@joinmacro{\pgf@nfold@roundjoin}
      \fi
    \fi
    % The last parameter is a macro to be called when this segment of the join is non-trivial.
    % It consists of
    %   - the macro to actually draw the join,
    %   - either finish@normal or finish@edgecase.
    \edef\pgf@nfold@macrotoadd{%
      \noexpand\pgf@nfold@token@join{\pgf@nfold@cur@first}{\pgf@nfold@prev@segment@end}
        {\pgf@nfold@cur@movedfirst}{\pgf@nfold@prev@angle@ii}{\pgf@nfold@cur@angle@i}%
        {\pgf@nfold@deltaphi@start}{\pgf@nfold@shortenstartjoin}{%
          \expandafter\noexpand\pgf@nfold@tmp@joinmacro%
          \expandafter\noexpand\ifpgf@nfold@closejoinsedgecase%
            \pgf@nfold@token@finish@edgecase{\pgf@nfold@cur@movedlast}%
          \else%
            \pgf@nfold@token@finish@normal%
          \fi%
        }%
    }%
    \pgf@nfold@addmacro\pgf@parsed@cur@conn@seg%
  \fi% end if abs(deltaphi) > 1
}


% All non-trivial joins connect to the end of the join, which is the starting point of the next segment.
% Note that this entire macro is skipped by if the start and end of the join coincide, so we never create a zero length segment here.
%
% There is one edge case here: If two subsequent joins are so close that \pgf@nfold@cur@movedlast
% and \pgf@nfold@cur@movedfirst exchange places *and* we are on the outside of the first join 
% (implying that we are on the inside of the second join), the first join must not connect all the way to
% to \pgf@nfold@cur@movedfirst, because that would overshoot the second join. Instead we connect to
% \pgf@nfold@cur@movedlast (which is *closer* to the first join than \pgf@nfold@cur@movedfirst in this edge case).
% To accomodate for the vertical offset we connect to \pgf@nfold@join@end which is the vertical
% offset of \pgf@nfold@cur@movedlast by definition.
\def\pgf@nfold@token@finish@normal{%
  \pgfpathlineto{\pgf@nfold@join@end}%
}

% parameter #1: @movedlast of the next segment
\def\pgf@nfold@token@finish@edgecase#1{%
  \pgf@nfold@join@end%
  \ifdim\insidepercentage pt<.5pt\relax%
    \pgf@process{\pgfpointadd{#1}{\pgfpointpolar{\pgf@nfold@secondang+90}{\pgf@nfold@shiftamount}}}
  \fi%
  \pgfpathlineto{}%
}


%
% Main rendering pipeline
% -----------------------
%

% This stores whether the current segment should begin with a moveto to its offset
\newif\ifpgf@nfold@continuesegment
% This stores whether we are in some edge case of very close joins, see below for details
\newif\ifpgf@nfold@closejoinsedgecase
% This stores whether we are in an error case where we need to avoid dividing by zero
\newif\ifpgf@nfold@angletoosharp
% This stores whether we need the intersections library for an arrow tip but it is not loaded
\newif\ifpgf@nfold@intersectionsnotloaded


\def\pgf@nfold@parser@handlesegment{%
  \if\pgf@nfold@cur@visible0
    % first, last and moveto are invisible
    \if\pgf@nfold@cur@type m
      % We don't need to do anything for a moveto: If a visible segment follows, it will move to
      % its starting location by itself. However, we might need to draw the arrow tip extension 
      % at the start (if present).
      \if\pgf@nfold@start@arrowcode1
        \if\pgf@nfold@prev@type f
          \if\pgf@nfold@next@visible1
            \edef\pgf@nfold@macrotoadd{%
              \noexpand\pgf@nfold@extendtotip{s}{\pgf@nfold@cur@last}{\pgf@nfold@next@angle@i}
            }%
            \pgf@nfold@addmacro\pgf@parsed@cur@conn@seg%
            % hack: We make the next segment believe that this segment was a lineto
            % so the path does not get interrupted
            \let\pgf@nfold@cur@type l
            \let\pgf@nfold@cur@visible 1
            \let\pgf@nfold@cur@angle@ii\pgf@nfold@next@angle@i
            \let\pgf@nfold@cur@tang@ii\pgf@nfold@next@tang@i
            \def\pgf@nfold@deltaphi@end{0}
            \pgf@nfold@angletoosharpfalse
          \fi
        \fi%
      \fi
    \fi%
  \else%
    \let\pgf@nfold@cur@movedfirst\pgf@nfold@cur@first
    \let\pgf@nfold@cur@movedlast\pgf@nfold@cur@last
    %%% Step 1: Make room for joins if necessary
    % In order to make room for the join, it may be necessary to shorten the current segment
    % at the start and/or the end. In here we store by how much the segment needs to be shortened.
    \def\pgf@nfold@shortenstartjoin{0}
    \def\pgf@nfold@shortenendjoin{0}
    \pgf@nfold@closejoinsedgecasefalse
    % Step 1.1: Make room for the join at the start if needed
    \if\pgf@nfold@prev@visible1
      \pgf@nfold@continuesegmenttrue
      % deltaphi@start can be recycled from deltaphi@end;
      % \ifpgf@nfold@angletoosharp is also still set
      \let\pgf@nfold@deltaphi@start\pgf@nfold@deltaphi@end
      % set \pgf@xb := abs(deltaphi@start)
      \pgf@xb=\pgf@nfold@deltaphi@start pt\relax
      \ifdim\pgf@xb<0pt\relax
        \pgf@xb=-\pgf@xb
      \fi
      \ifpgf@nfold@angletoosharp\else
        \ifdim\pgf@xb>0.5pt\relax
          % make room for the start join if the angle is nonzero;
          % shortenstartjoin := hwidth*tan(.5*abs(deltaphi@start))
          \pgf@yb=.5\pgf@xb
          \pgfmathtan@{\pgf@sys@tonumber\pgf@yb}
          \pgf@yb=\pgf@nfold@hwidth\relax
          \pgf@yb=\pgfmathresult\pgf@yb
          \edef\pgf@nfold@shortenstartjoin{\pgf@sys@tonumber\pgf@yb}
          \pgfextract@process\pgf@nfold@cur@movedfirst{%
            \pgfpointadd{\pgf@nfold@cur@first}%
                        {\pgfqpointscale{\pgf@nfold@shortenstartjoin}{\pgf@nfold@cur@tang@i}}}%
          % If the current segment is a curve, we need to relocate @supporta point as well,
          % as otherwise the @first point could overtake it
          \if\pgf@nfold@cur@type c
            \pgf@process{\pgfpointdiff{\pgf@nfold@cur@first}{\pgf@nfold@cur@supporta}}
            \pgfmathveclen@{\pgf@sys@tonumber\pgf@x}{\pgf@sys@tonumber\pgf@y}
            \ifdim\pgfmathresult pt>0.1pt\relax
              % regular curves (supporta != first):
              % Increase dist(first, supporta) to sqrt(a^2 + b^2) where a=dist(first, supporta) and b=shortenstart. This way, the order of first and supporta is guaranteed to be preserved, and the change to supporta is as small as reasonably possible.
              \pgfmathveclen@{\pgfmathresult}{\pgf@sys@tonumber\pgf@yb}
              \pgfextract@process\pgf@nfold@cur@supporta{\pgfpointadd%
                {\pgf@nfold@cur@first}%
                {\pgfqpointscale{\pgfmathresult}{\pgf@nfold@cur@tang@i}}}%
            \else
              % special treatment for singular curves (supporta = first) to avoid rounding error glitches.
              % In this special case, a slight corner at the end of the join is unavoidable unless we
              % also relocate @supportb, which may have unintended side effects
              \let\pgf@nfold@cur@supporta\pgf@nfold@cur@movedfirst
            \fi
          \fi
        \fi
      \fi
    \else
      \pgf@nfold@continuesegmentfalse
    \fi
    % Step 1.2: Make room for the join at the end if needed
    \if\pgf@nfold@next@visible1
      % Compute the angle difference at the start (between -180 and +180 degrees)
      % using \pgfmathsubtract@ is more readable and no less efficient than computing this manually
      \pgfmathsubtract@{\pgf@nfold@next@angle@i}{\pgf@nfold@cur@angle@ii}
      \pgf@nfold@clampangle
      \edef\pgf@nfold@deltaphi@end{\pgfmathresult}
      \pgf@xb=\pgf@nfold@deltaphi@end pt\relax
      \ifdim\pgf@xb<0pt\relax
        \pgf@xb=-\pgf@xb
      \fi
      \ifdim\pgf@xb>178pt\relax
        \pgfutil@packagewarning{tikz-nfold}{Angle too sharp, expect visual errors}
        \pgf@nfold@angletoosharptrue
      \else
        \pgf@nfold@angletoosharpfalse
        \ifdim\pgf@xb>0.5pt\relax
           % make room for the start join if the angle is nonzero
          % shortenendjoin := hwidth*tan(.5*abs(deltaphi@end))
          \pgf@yb=.5\pgf@xb
          \pgfmathtan@{\pgf@sys@tonumber\pgf@yb}
          \pgf@yb=\pgf@nfold@hwidth\relax
          \pgf@yb=\pgfmathresult\pgf@yb
          \edef\pgf@nfold@shortenendjoin{\pgf@sys@tonumber\pgf@yb}
          \pgf@yb=-\pgf@yb
          \pgfextract@process\pgf@nfold@cur@movedlast{%
            % the use of the minus sign is fine here because \pgf@nfold@shortenendjoin >= 0.0
            \pgfpointadd{\pgf@nfold@cur@last}%
              {\pgfqpointscale{-\pgf@nfold@shortenendjoin}{\pgf@nfold@cur@tang@ii}}}%
          \pgfextract@process\pgf@nfold@cur@movedlast{%
            \pgfpointadd{\pgf@nfold@cur@last}%
              {\pgfqpointpolar{\pgf@nfold@cur@angle@ii}{\pgf@yb}}}%
          \if\pgf@nfold@cur@type c
            % Same procedure as above: relocate supportb if we have a curve
            \pgf@process{\pgfpointdiff{\pgf@nfold@cur@supportb}{\pgf@nfold@cur@last}}
            \pgfmathveclen@{\pgf@sys@tonumber\pgf@x}{\pgf@sys@tonumber\pgf@y}
            \ifdim\pgfmathresult pt>0.1pt\relax
              \pgfmathveclen@{\pgfmathresult}{\pgf@sys@tonumber\pgf@yb}
              \pgfextract@process\pgf@nfold@cur@supportb{\pgfpointadd%
                {\pgf@nfold@cur@last}%
                % can use qpointscale and a minus because \pgfmathresult is guaranteed to be positive
                {\pgfqpointscale{-\pgfmathresult}{\pgf@nfold@cur@tang@ii}}}%
            \else
              \let\pgf@nfold@cur@supportb\pgf@nfold@cur@movedlast
            \fi
          \fi
        \fi
      \fi
      % Step 1.3: Detect an edge case
      % This edge case appears whenever the current segment is such a short line that we would
      % have to reduce its length to less than zero to make space for the joins. In such cases,
      % the line is not drawn at all, and slight modifications must be made to the joins to ensure
      % a correct output (i.e. one join is immediately followed by the next without a segment in between).
      %
      % This edge case can appear for curves as well, but they are much harder to deal with.
      \if\pgf@nfold@cur@type l
        % No need to check for \pgf@nfold@inputsegmentclosepath as it should not be followed by any further segments
        % Now: Check if the total amount of shortening is larger than the length of the segment
        \pgf@process{\pgfpointdiff{\pgf@nfold@cur@first}{\pgf@nfold@cur@last}}
        \pgfmathveclen@{\pgf@sys@tonumber\pgf@x}{\pgf@sys@tonumber\pgf@y}
        \pgf@xa=\pgf@nfold@shortenstartjoin pt\relax
        \advance\pgf@xa by\pgf@nfold@shortenendjoin pt\relax
        \ifdim\pgf@xa>\pgfmathresult pt\relax
          \pgf@nfold@closejoinsedgecasetrue
        \fi
      \fi
    \fi% end if next segment visible
    %
    % Step 2.1: Draw the join at the start if applicable
    %
    \if\pgf@nfold@prev@visible0%
      \ifpgf@nfold@closejoinsedgecase
        % If the previous segment is a moveto and the current segment is a "close joins" edge case,
        % nothing needs to be drawn here (the relevant draw call will be made at the join of the subsequent
        % segment). We must therefore make sure that we move to the correct end point of this segment.
        % Counterintuitively, this is given by the offset of \pgf@nfold@cur@movedfirst since the start and end
        % are reversed in the edge case.
        \edef\pgf@nfold@macrotoadd{%
          \noexpand\pgf@nfold@token@edgecase@movetostart{\pgf@nfold@cur@movedfirst}{\pgf@nfold@cur@angle@i}%
        }%
        \pgf@nfold@addmacro\pgf@parsed@cur@conn@seg%
      \fi
    \else
      % If we draw the join when the start angle is close to 180 degrees, we get a division by zero
      \ifpgf@nfold@angletoosharp\else
        \pgf@nfold@parser@handlejoin
      \fi
    \fi
    % Step 2.2: Store where the current (non-offset) end point was relocated
    % in order to make space for the end join. This may be used if the next
    % segment begins with a join
    \let\pgf@nfold@prev@segment@end\pgf@nfold@cur@movedlast
    %
    % Step 3: Draw the new segment.
    %
    % The value of \ifpgf@nfold@continuesegment decides whether we start with a moveto.
    % curveto
    \if\pgf@nfold@cur@type l
      % In the edge case, one join is followed immediately by the next. The line segment
      % thus has a negative length and will be skipped.
      \ifpgf@nfold@closejoinsedgecase\else
        \edef\pgf@nfold@macrotoadd{%
          \expandafter\noexpand\ifpgf@nfold@continuesegment%
            \pgf@nfold@token@lineto@continue%
          \else%
            \pgf@nfold@token@lineto%
          \fi{\pgf@nfold@cur@movedfirst}{\pgf@nfold@cur@movedlast}{\pgf@nfold@cur@tang@i}%
        }%
        \pgf@nfold@addmacro\pgf@parsed@cur@conn@seg%
      \fi
    \fi
    \if\pgf@nfold@cur@type o
      \def\pgf@nfold@macrotoadd{\pgf@nfold@token@closepath}%
      \pgf@nfold@addmacro\pgf@parsed@cur@conn@seg%
    \fi
    \if\pgf@nfold@cur@type c
      \ifpgf@nfold@continuesegment
        \pgf@subdividecurve{\pgf@nfold@cur@movedfirst}{\pgf@nfold@cur@supporta}{\pgf@nfold@cur@supportb}{\pgf@nfold@cur@movedlast}{\pgf@offset@max@recursion}{0}{\pgf@nfold@addcurvesegment@callback@continue}
      \else
        \pgf@subdividecurve{\pgf@nfold@cur@movedfirst}{\pgf@nfold@cur@supporta}{\pgf@nfold@cur@supportb}{\pgf@nfold@cur@movedlast}{\pgf@offset@max@recursion}{0}{\pgf@nfold@addcurvesegment@callback}%
      \fi
    \fi
    \if\pgf@nfold@cur@type i
      \edef\pgf@nfold@macrotoadd{%
        \noexpand\pgf@nfold@token@invisibleline{\pgf@nfold@cur@movedfirst}{\pgf@nfold@cur@movedlast}{\pgf@nfold@cur@tang@i}%
      }%
      \pgf@nfold@addmacro\pgf@parsed@cur@conn@seg%
    \fi
    % Step 4: Extend into the arrow tip at the end (if present)
    \if\pgf@nfold@next@type t%
      \ifnum\pgf@nfold@end@arrowcode=1
        \edef\pgf@nfold@macrotoadd{%
          \noexpand\pgf@nfold@extendtotip{e}{\pgf@nfold@cur@last}{\pgf@nfold@cur@angle@ii}}%
        \pgf@nfold@addmacro\pgf@parsed@cur@conn@seg%
      \fi%
    \fi%
  \fi% end if current visible
}


\def\pgf@nfold@addcurvesegment@callback#1#2#3#4#5{%
  \if#50%
    \edef\pgf@nfold@macrotoadd{%
      % The subdivision algorithm has already computed the tangents
      \noexpand\pgf@nfold@token@curveto{#1}{#2}{#3}{#4}{\pgf@tmp@tang@i}{\pgf@tmp@tang@ii}%
    }%
    \pgf@nfold@addmacro\pgf@parsed@cur@conn@seg%
  \else%
    \pgf@nfold@addcurvesegment@callback@continue{#1}{#2}{#3}{#4}{#5}%
  \fi%
}

\def\pgf@nfold@addcurvesegment@callback@continue#1#2#3#4#5{%
  % The subdivision algorithm has already computed the tangents
  \edef\pgf@nfold@macrotoadd{%
    \noexpand\pgf@nfold@token@curveto@continue{#1}{#2}{#3}{#4}{\pgf@tmp@tang@i}{\pgf@tmp@tang@ii}%
  }%
  \pgf@nfold@addmacro\pgf@parsed@cur@conn@seg%
}


% Rendering arrow tips
% --------------------

% Precomputed intersections
%
% For arrows of order n > 2 with an Implies tip, the constituent parts of the n-fold arrow
% end somewhere in the middle of the tip. The exact end point must be computed using
% the intersections library. To speed up compilation times, the intersection points are precomputed
% up to n = 5. If your document contains arrows of order 6 or larger, consider adding those
% as well; the values are output in the log file.
\expandafter\def\csname pgf@nfold@intersec@cache@2@3\endcsname{\pgfqpoint{2pt}{0pt}}
\expandafter\def\csname pgf@nfold@intersec@cache@2@4\endcsname{\pgfqpoint{0.94063pt}{-0.33333pt}}
\expandafter\def\csname pgf@nfold@intersec@cache@3@4\endcsname{\pgfqpoint{0.94063pt}{0.33333pt}}
\expandafter\def\csname pgf@nfold@intersec@cache@2@5\endcsname{\pgfqpoint{0.64167pt}{-0.5pt}}
\expandafter\def\csname pgf@nfold@intersec@cache@3@5\endcsname{\pgfqpoint{2pt}{0pt}}
\expandafter\def\csname pgf@nfold@intersec@cache@4@5\endcsname{\pgfqpoint{0.64167pt}{0.5pt}}
% intersections are precomputed up to this order
\def\pgf@nfold@intersec@numcached{5}

% This macro extends the arrow body to the tips
% #1: s=start, e=end
% #2: start/end point of the path
% #3: angle
\newif\ifpgf@nfold@ontheedge
\def\pgf@nfold@extendtotip#1#2#3{
  \ifpgf@nfold@intersectionsnotloaded
    \pgfutil@packageerror{tikz-nfold}{%
        If `nfold' is larger than \pgf@nfold@intersec@numcached\space and you use
        an `Implies' arrow tip you need to say \string\usetikzlibrary{intersections}}{}
  \else
  \pgf@nfold@ontheedgetrue
  % Do not extend the arrow for index=1 and index=order, it already ends in the right place
  \ifnum\pgf@nfold@index>1\relax\ifnum\pgf@nfold@index<\pgf@nfold@order\relax%
    \pgf@nfold@ontheedgefalse
  \fi\fi
  \ifpgf@nfold@ontheedge%
    % the extension at the start needs a moveto to the correct starting point
    \if#1s
      \pgfpathmoveto{\pgfpointadd{#2}{\pgfpointpolar{#3+90}{\pgf@nfold@shiftamount}}}%
    \fi
  \else
    % Step 1: Find the intersection of the arrow's path with the head
    \ifcsname pgf@nfold@intersec@cache@\the\pgf@nfold@index @\the\pgf@nfold@order\endcsname
      \pgfextract@process\pgf@nfold@arrowintersect
        {\csname pgf@nfold@intersec@cache@\the\pgf@nfold@index @\the\pgf@nfold@order\endcsname}%
    \else
      % the intersection has not been precomputed, thus compute on the fly here
      \pgfintersectionofpaths{
        % specify the tip
        \pgfpathmoveto{\pgfqpoint{-1.4pt}{2.65pt}}
        \pgfpathcurveto{\pgfqpoint{-0.75pt}{1.25pt}}{\pgfqpoint{1pt}{0.05pt}}{\pgfqpoint{2pt}{0pt}}
        \pgfpathcurveto{\pgfqpoint{1pt}{-0.05pt}}{\pgfqpoint{-0.75pt}{-1.25pt}}{\pgfqpoint{-1.4pt}{-2.65pt}}
      }{
        % extend the body to intersect the tip
        \pgfpathmoveto{\pgfqpoint{-3pt}{\pgf@nfold@shift@fraction pt}}
        \pgfpathlineto{\pgfqpoint{3pt}{\pgf@nfold@shift@fraction pt}}
      }
      \ifnum\pgfintersectionsolutions>0
        \pgfextract@process\pgf@nfold@arrowintersect{\pgfpointintersectionsolution{1}}%
        \immediate\write17{tikz-nfold: computed intersection cache@\the\pgf@nfold@index @\the\pgf@nfold@order: \string\pgfqpoint{\the\pgf@x}{\the\pgf@y}^^J}
        % add the new intersection to the cache
        \expandafter\xdef\csname pgf@nfold@intersec@cache@\the\pgf@nfold@index @\the\pgf@nfold@order\endcsname{\noexpand\pgfqpoint{\the\pgf@x}{\the\pgf@y}}
      \else
        % this is a failsafe and should never be reached
        \pgfutil@packagewarning{tikz-nfold}{did not find intersection}
        \pgfextract@process\pgf@nfold@arrowintersect{\pgfqpoint{0pt}{\pgf@nfold@shift@fraction pt}}%
      \fi
    \fi% if precomputed
    % Step 2: Extend the arrow body to the intersection point.
    % If the tip is at the beginning of the path, we have to move to the intersection
    % and then draw a line to the "regular" starting point. The subsequent segment then
    % should omit its moveto.
    % If the tip is at the end, we are already in the right position and only need to extend
    % the current path to the intersection point.
    \begingroup
      \pgftransformreset
      \pgftransformshift{#2}
      \pgftransformrotate{#3}
      \if#1s
        \pgftransformxscale{-1}
      \fi
      % we don't want to undo the shift by .42\pgflinewidth after the scaling
      \pgfutil@tempdima=\pgf@nfold@hwidth
      \pgfutil@tempdima=\pgf@nfold@shift@fraction\pgfutil@tempdima
      \pgfextract@process\pgf@nfold@startofextension
        {\pgfpointtransformed{\pgfqpoint{0pt}{\pgfutil@tempdima}}}
      % 0.5 - 0.06 = 0.42
      \pgftransformshift{\pgfqpoint{.42\pgflinewidth}{0pt}}
      \pgftransformscale{\pgf@nfold@hwidth}
      \pgfextract@process\pgf@nfold@arrowintersect{\pgfpointtransformed{\pgf@nfold@arrowintersect}}
      \global\let\pgf@nfold@startofextension\pgf@nfold@startofextension
      \global\let\pgf@nfold@arrowintersect\pgf@nfold@arrowintersect
    \endgroup
    \if#1s
      \pgfpathmoveto{\pgf@nfold@arrowintersect}
      % This is precisely the start of the body, shifted vertically
      \pgfpathlineto{\pgf@nfold@startofextension}
    \else\if#1e
      \pgfpathlineto{\pgf@nfold@arrowintersect}
    \else
      \pgfutil@packageerror{tikz-nfold}{Invalid argument to \string\pgf@nfold@extendtotip: \meaning#1}{}
    \fi\fi
    \pgftransformreset
    \fi% if 1 < i < nArrows
  \fi% if intersections is needed and not loaded
}

% Parsing the arrow tips
% ----------------------
%
% We need to detect if the user has set Implies[] arrows at the start and/or end tip.
% To do so, we parse \pgf@start@tip@sequence. If the user specifies Implies[] manually,
% we find that
%   pgf@start@tip@sequence=\pgf@arrow@handle{Implies}{...}
% However, in other cases (like tikz-cd) we may find 
%   \pgf@arrow@handle@shorthand@empty {\csname pgf@ar@means@tikzcd implies cap\endcsname }
% In such cases we must expand the first parameter once and then match as above.

% Set global defaults
\def\pgf@nfold@start@arrowcode{0}
\def\pgf@nfold@end@arrowcode{0}

\def\pgf@nfold@parsearrows{
  \ifpgfutil@tempswa% this is set in \pgfusepath and stores whether we draw arrow tips at all
    \expandafter\pgf@nfold@parsearrowmacro\pgf@start@tip@sequence\relax
    \let\pgf@nfold@start@arrowcode\pgf@nfold@detectedarrow
    \expandafter\pgf@nfold@parsearrowmacro\pgf@end@tip@sequence\relax
    \let\pgf@nfold@end@arrowcode\pgf@nfold@detectedarrow
  \else
    \def\pgf@nfold@start@arrowcode{0}
    \def\pgf@nfold@end@arrowcode{0}
  \fi
}

\def\pgf@nfold@parsearrowmacro#1{%
  \def\pgf@nfold@detectedarrow{0}
  \ifx#1\relax
    \let\pgf@next\relax
  \else
    \ifx#1\pgf@arrow@handle
      % found \pgf@arrow@handle{...}, now parse the first parameter
      \let\pgf@next\pgf@nfold@parse@arrow@handle
    \else
      \ifx#1\pgf@arrow@handle@shorthand@empty
        \let\pgf@next\pgf@nfold@parse@shorthandempty
      \else
        % found nothing
        \let\pgf@next\pgfutil@gobble@until@relax
      \fi
    \fi
  \fi
  \pgf@next
}

\def\pgf@nfold@param@Implies{Implies}

\def\pgf@nfold@parse@arrow@handle#1{%
  \def\pgf@tmp{#1}
  \ifx\pgf@tmp\pgf@nfold@param@Implies
    \def\pgf@nfold@detectedarrow{1}
  \fi
  \pgfutil@gobble@until@relax
}

\def\pgf@nfold@parse@shorthandempty#1{
  % Expand #1 once (\pgf@arrow@handle@shorthand@empty is just an identity operator)
  \expandafter\def\expandafter\pgf@tmp\expandafter{#1}
  \expandafter\pgf@nfold@parsearrowmacro\pgf@tmp\relax
  % still need to gobble the rest of the orginal arrow definition
  \pgfutil@gobble@until@relax
}


%
% Hooking into pgf's rendering pipeline
% -------------------------------------
%
% The new code has to be injected into \pgfusepath (pgfcorepathusage.code.tex). For rendering the new paths,
% \pgf@stroke@inner@line is a natural choice as this is where /tikz/double is rendered. However, we also
% need to disable rendering the ordinary path, which is not as easy. In the future I will make a pull request
% to TikZ to simplify such injections.
%
% The call to draw the path comes right before \pgf@stroke@inner@line. The macro before \pgf@stroke@inner@line
% is either \pgf@path@check@proper or \pgf@prepare@start@of@path (depending on the result of the proper check).
% We therefore must inject code into both of them to see if nfold is enabled. If it is, we call the old macro,
% cache and delete the current softpath (so the call to \pgfsyssoftpath@invokecurrentpath has no effect), then we
% restore and offset the cached softpath in \pgf@stroke@inner@line.
%
% The macros \pgf@path@check@proper and \pgf@prepare@start@of@path are also used in \pgf@up@draw@arrows@only,
% so we must make sure that the latter is unaffected by the modifications. Luckily, this turns out not to be
% a problem - the only macros that are called after the modified ones are \pgf@add@arrow@at@start and
% \pgf@add@arrow@at@end, which do not change their behaviour even if we modify the paths.
%

\newcount\pgf@nfold@order
\pgf@nfold@order=1

\def\pgf@nfold@preparenfoldpath{%
  \ifnum\pgf@nfold@order>1\relax
    \ifdim\pgfinnerlinewidth>0pt\relax
      % Hack the rendering pipeline: There is a \pgfsyssoftpath@invokecurrentpath call following
      % which we do not want if nfold is active. We therefore clear the current path here
      % and then perform the nfold drawing in our modification of \pgf@stroke@inner@line
      \pgfsyssoftpath@getcurrentpath\pgf@nfold@cachedpath%
      \pgfsyssoftpath@setcurrentpath\pgfutil@empty%
    \else
      \pgfutil@packageerror{tikz-nfold}{Must set \string\pgfinnerlinewidth\space to use nfold, e.g. by setting /tikz/double distance}{}
    \fi
  \fi
}

\let\pgf@nfold@old@path@check@proper\pgf@path@check@proper
\def\pgf@path@check@proper{%
  \pgf@nfold@old@path@check@proper%
  \ifpgfutil@tempswa\else%
    % if \pgfutil@tempswa is false, this is the last macro we can overwrite before the draw call.
    % Otherwise, we inject into \pgf@prepare@start@of@path%
    \pgf@nfold@preparenfoldpath%
  \fi%
}

\let\pgf@nfold@old@prepare@start@of@path\pgf@prepare@start@of@path
\def\pgf@prepare@start@of@path{%
  \pgf@nfold@old@prepare@start@of@path%
  \pgf@nfold@preparenfoldpath%
}

\let\pgf@nfold@old@stroke@inner@line\pgf@stroke@inner@line
\def\pgf@stroke@inner@line{%
  \ifnum\pgf@nfold@order>1\relax%
    \pgf@nfold@render@cached@softpath%
  \else%
    % Old behaviour
    \pgf@nfold@old@stroke@inner@line%
  \fi%
}


%
% Parsing the soft path
% ---------------------
%
% A significant part of the code below is based on pgfmoduledecorations.code.tex (c) 2019 Mark Wibrow and Till Tantau.
% Quite similar to decorations we parse the current soft path and put it into a form that makes it easier
% to iterate over.
%
%


%
% In order to correctly implement \pgfpathclose we already need to know about
% the \pgfpathclose (and the penultimate point) at the beginning of this
% connected segment as these data affect the first/last join.
% We therefore parse one connected segment of the softpath and store it in a modified
% form in \pgf@cur@conn@segment. When reaching the end of the connected segment we call
% the proper parser to turn \pgf@cur@conn@segment into a parsed path. We do this for
% all connected segments and concatenate all the parsed paths.
%


\def\pgf@nfold@parsesoftpath#1#2{%
  \let\pgf@cur@conn@segment\pgfutil@empty%
  \let\pgf@all@parsed@segments\pgfutil@empty%
  \edef\pgf@nfold@parser@last@moveto{{\the\pgf@path@lastx}{\the\pgf@path@lasty}}%
  \let\pgf@nfold@last@closepath@from\pgfutil@empty%
  \pgf@nfold@parse@setupfirst%
  \expandafter\pgf@nfold@@parsesoftpath#1\pgf@stop{}{}%
  \let#2\pgf@all@parsed@segments%
}%

\def\pgf@nfold@@parsesoftpath#1#2#3{%
  \let\pgf@next\pgf@nfold@@parsesoftpath%
  \ifx#1\pgf@stop%
    \def\pgf@nfold@macrotoadd{\pgf@nfold@parselast}%
    \pgf@nfold@addmacro\pgf@cur@conn@segment%
    \pgf@nfold@process@conn@segment%
    \let\pgf@next\relax%
  \else%
    \ifx#1\pgfsyssoftpath@movetotoken%
      \ifx\pgf@cur@conn@segment\pgfutil@empty%
        % This case happens for the very first segment or for double movetos.
        % We need special treatment here, as otherwise the arrow tip extension
        % does not work correctly. The \pgf@nfold@parsemoveto will be called
        % in \pgf@nfold@process@conn@segment.
        \def\pgf@nfold@parser@last@moveto{{#2}{#3}}%
        % Make sure \pgf@cur@conn@segment is no longer empty so double movetos are not
        % treated the same as single movetos. This is also relevant to arrow tip extensions
        \def\pgf@cur@conn@segment{\relax}%
      \else%
        \def\pgf@nfold@macrotoadd{\pgf@nfold@parsemoveto{#2}{#3}}%
        \pgf@nfold@addmacro\pgf@cur@conn@segment%
        % A moveto marks the beginning/end of one connected segment
        \pgf@nfold@process@conn@segment%
        \let\pgf@nfold@last@closepath@from\pgfutil@empty%
        \let\pgf@cur@conn@segment\pgfutil@empty%
        \def\pgf@nfold@parser@last@moveto{{#2}{#3}}%
      \fi
    \else%
      \ifx#1\pgfsyssoftpath@linetotoken%
        \def\pgf@nfold@macrotoadd{\pgf@nfold@parselineto{#2}{#3}}%
        \pgf@nfold@addmacro\pgf@cur@conn@segment%
      \else%
        \ifx#1\pgfsyssoftpath@curvetosupportatoken%
          \def\pgf@nfold@parse@supporta{{#2}{#3}}%
        \else%
          \ifx#1\pgfsyssoftpath@curvetosupportbtoken%
            \def\pgf@nfold@parse@supportb{{#2}{#3}}%
          \else%
            \ifx#1\pgfsyssoftpath@curvetotoken%
              \edef\pgf@nfold@macrotoadd{\noexpand\pgf@nfold@parsecurveto\pgf@nfold@parse@supporta\pgf@nfold@parse@supportb{#2}{#3}}%
              \pgf@nfold@addmacro\pgf@cur@conn@segment%
            \else%
              \ifx#1\pgfsyssoftpath@closepathtoken%
                \let\pgf@nfold@last@closepath@from\pgf@nfold@parser@previous@pt%
                \def\pgf@nfold@macrotoadd{\pgf@nfold@parseclosepath{#2}{#3}}%
                \pgf@nfold@addmacro\pgf@cur@conn@segment%
              \else%
                \ifx#1\pgfsyssoftpath@rectcornertoken%
                  \def\pgf@nfold@parse@rectcorner{{#2}{#3}}%
                \else%
                  \ifx#1\pgfsyssoftpath@rectsizetoken%
                    \edef\pgf@nfold@macrotoadd{\noexpand\pgf@nfold@parserect\pgf@nfold@parse@rectcorner{#2}{#3}}%
                    \pgf@nfold@addmacro\pgf@cur@conn@segment%
                  \else%
                    \pgfutil@packageerror{tikz-nfold}{Unrecognised soft path token `#1'}{}%
                  \fi%
                \fi%
              \fi%
            \fi%
          \fi%
        \fi%
      \fi%
    \fi%
  \fi%
  \def\pgf@nfold@parser@previous@pt{{#2}{#3}}%
  \pgf@next}%

\def\pgf@nfold@process@conn@segment{%
  \let\pgf@parsed@cur@conn@seg\pgfutil@empty%
  \ifx\pgf@nfold@last@closepath@from\pgfutil@empty%
    \expandafter\pgf@nfold@parsemoveto\pgf@nfold@parser@last@moveto%
  \else%
    % This connected segment ends on a closepath. In order to get
    % the join right, we prepend the current segment with an invisible line
    % identical to the line of the \pgfpathclose. This way, 
    \expandafter\pgf@nfold@parsemoveto\pgf@nfold@last@closepath@from%
    \expandafter\pgf@nfold@parseinvisibleline\pgf@nfold@parser@last@moveto%
  \fi%
  \pgf@cur@conn@segment%
  \let\pgf@nfold@macrotoadd\pgf@parsed@cur@conn@seg%
  \pgf@nfold@addmacro\pgf@all@parsed@segments%
}

%
% Values for \pgf@nfold@next@type:
% f=first, m=moveto, c=curveto, l=lineto, r=rect, o=closepath, t=last, i=invisibleline
%

\def\pgf@nfold@parsemoveto#1#2{%
  \pgf@nfold@parser@moveup%
  \let\pgf@nfold@next@type=m%
  \let\pgf@nfold@next@visible=0%
  \edef\pgf@nfold@next@last{\pgf@x#1\pgf@y#2}%
  \pgf@nfold@parser@handlesegment%
}%


% Common code for parselineto and parseclosepath
\def\pgf@nfold@parse@line@common#1#2#3{%
  \pgf@nfold@parser@moveup%
  \pgfextract@process\pgf@tmp@tang@i{\pgfpointnormalised{}\global\let\pgf@nfold@tmp\pgf@tmp}
  \let\pgf@nfold@next@type=#3%
  \let\pgf@nfold@next@visible=1%
  \let\pgf@nfold@next@tang@i\pgf@tmp@tang@i
  \let\pgf@nfold@next@tang@ii\pgf@tmp@tang@i
  \let\pgf@nfold@next@angle@i\pgf@nfold@tmp
  \let\pgf@nfold@next@angle@ii\pgf@nfold@tmp
  \let\pgf@nfold@next@first\pgf@nfold@cur@last
  \def\pgf@nfold@next@last{\pgf@x#1\pgf@y#2}%
  \pgf@nfold@parser@handlesegment%
}

\def\pgf@nfold@parselineto#1#2{%
  % the current end is still stored in next@last because parser@moveup has not been called yet
  \pgf@process{\pgfpointdiff{\pgf@nfold@next@last}{\pgf@x#1\pgf@y#2}}%
  \pgfpointtaxicabnorm\pgf@xa%
  % remove degenerate line segments (reduces glitches)
  \ifdim\pgf@xa>.1pt\relax%
    \pgf@nfold@parse@line@common{#1}{#2}{l}%
  \fi%
}%

\def\pgf@nfold@parsecurveto#1#2#3#4#5#6{%
  \pgf@nfold@parser@moveup%
  \pgf@offset@compute@tangents{\pgf@nfold@cur@last}{\pgf@x#1\pgf@y#2}{\pgf@x#3\pgf@y#4}{\pgf@x#5\pgf@y#6}%
  \let\pgf@nfold@next@type=c%
  \let\pgf@nfold@next@visible=1%
  % the following only need to be set if visible=1
  \let\pgf@nfold@next@tang@i=\pgf@tmp@tang@i
  \let\pgf@nfold@next@tang@ii=\pgf@tmp@tang@ii
  \let\pgf@nfold@next@angle@i\pgf@tmp@angle@i
  \let\pgf@nfold@next@angle@ii\pgf@tmp@angle@ii
  \let\pgf@nfold@next@first\pgf@nfold@cur@last
  \def\pgf@nfold@next@supporta{\pgf@x#1\pgf@y#2}
  \def\pgf@nfold@next@supportb{\pgf@x#3\pgf@y#4}
  \def\pgf@nfold@next@last{\pgf@x#5\pgf@y#6}
  \pgf@nfold@parser@handlesegment%
}%

\def\pgf@nfold@parseclosepath#1#2{%
  \pgf@process{\pgfpointdiff{\pgf@nfold@next@last}{\pgf@x#1\pgf@y#2}}%
  % closepath segments should always be processed even if they have zero length
  \pgf@nfold@parse@line@common{#1}{#2}{o}%
}%

\def\pgf@nfold@parseinvisibleline#1#2{%
  \pgf@process{\pgfpointdiff{\pgf@nfold@next@last}{\pgf@x#1\pgf@y#2}}%
  % invisibleline segments should always be processed even if they have zero length.
  % Counterintuitively, an "invisibleline" segment has next@visible=1 because
  % we want its end join to be rendered.
  \pgf@nfold@parse@line@common{#1}{#2}{i}%
}%

\def\pgf@nfold@parselast{%
  \pgf@nfold@parser@moveup%
  \let\pgf@nfold@next@type=t%
  \let\pgf@nfold@next@visible=0%
  \pgf@nfold@parser@handlesegment%
}

% Mostly for the sake of completeness; using TikZ' "\path (0,0) rectangle (1,1);" does not call this code
\def\pgf@nfold@parserect#1#2#3#4{%
  \pgf@nfold@parsemoveto{#1}{#2}%
  \pgf@xc=#1\relax
  \pgf@yc=#2\relax
  \pgf@xd=#3\relax
  \pgf@yd=#4\relax
  \advance\pgf@yc\pgf@yd%
  \edef\pgf@temp{{\the\pgf@xc}{\the\pgf@yc}}%
  \expandafter\pgf@nfold@parselineto\pgf@temp%
  \advance\pgf@xc\pgf@xd%
  \edef\pgf@temp{{\the\pgf@xc}{\the\pgf@yc}}%
  \expandafter\pgf@nfold@parselineto\pgf@temp%
  \advance\pgf@yc-\pgf@yd%
  \edef\pgf@temp{{\the\pgf@xc}{\the\pgf@yc}}%
  \expandafter\pgf@nfold@parselineto\pgf@temp%
  \advance\pgf@xc-\pgf@xd%
  \edef\pgf@temp{{\the\pgf@xc}{\the\pgf@yc}}%
  \expandafter\pgf@nfold@parselineto\pgf@temp%
  \edef\pgf@marshal{\noexpand\pgf@nfold@parsemoveto{\the\pgf@xc}{\the\pgf@yc}}%
  \pgf@marshal%
}

% adds the contents of \pgf@nfold@macrotoadd to #1
\def\pgf@nfold@addmacro#1{%
  % need a \gdef because we have nested groups in the subdivision
  \expandafter\expandafter\expandafter\gdef%
  \expandafter\expandafter\expandafter#1%
  \expandafter\expandafter\expandafter{%
  \expandafter#1\pgf@nfold@macrotoadd}%
}


%
% Tokens for the parsed path
%

%#3: cached normalised tangent (for efficiency)
\def\pgf@nfold@token@lineto#1#2#3{%
  \pgfoffsetline@{#1}{#2}{\pgf@nfold@shiftamount}{#3}%
}

\def\pgf@nfold@token@lineto@continue#1#2#3{%
  \pgfoffsetlinenomove@{#1}{#2}{\pgf@nfold@shiftamount}{#3}%
}

\def\pgf@nfold@token@invisibleline#1#2#3{%
  % An invisible line should execute a moveto to the end point of the line
  \pgfqpointscale{\pgf@nfold@shiftamount}{#3}%
  \pgf@xc=-\pgf@y%
  \pgf@y=\pgf@x%
  \pgf@x=\pgf@xc%
  \pgfpathmoveto{\pgfpointadd{}{#2}}%
}

% #5, #6: cached normalised tangents at start and end (for efficiency)
\def\pgf@nfold@token@curveto#1#2#3#4#5#6{%
  \def\pgf@tmp@tang@i{#5}%
  \def\pgf@tmp@tang@ii{#6}%
  \pgf@offset@bezier@segment@{#1}{#2}{#3}{#4}{\pgf@nfold@shiftamount}%
  \pgfpathmoveto{\pgf@bezier@offset@i}%
  \pgfpathcurveto{\pgf@bezier@offset@ii}{\pgf@bezier@offset@iii}{\pgf@bezier@offset@iv}%
}

\def\pgf@nfold@token@curveto@continue#1#2#3#4#5#6{%
  \def\pgf@tmp@tang@i{#5}%
  \def\pgf@tmp@tang@ii{#6}%
  \pgf@offset@bezier@segment@{#1}{#2}{#3}{#4}{\pgf@nfold@shiftamount}%
  \pgfpathcurveto{\pgf@bezier@offset@ii}{\pgf@bezier@offset@iii}{\pgf@bezier@offset@iv}%
}

\def\pgf@nfold@token@closepath{%
  \pgfpathclose%
}

% This is needed for some edge case explained above.
% #1: segment@start, which is actually the end point of the current segment in the edge case
% #2: the angle of the current segment (start or end does not matter for straight lines)
\def\pgf@nfold@token@edgecase@movetostart#1#2{%
  \pgfpathmoveto{\pgfpointadd%
    {#1}{\pgfpointpolar{#2+90}{\pgf@nfold@shiftamount}}}%
}


\def\pgf@nfold@parse@setupfirst{%
  \let\pgf@nfold@next@type=f%
  % technically, @visible can be derived from @type, but I think this is more efficient
  \let\pgf@nfold@next@visible=0%
  \edef\pgf@nfold@next@last{\pgf@x\the\pgf@path@lastx\pgf@y\the\pgf@path@lasty}%
  % the following only need to be set if visible=1
  \let\pgf@nfold@next@tang@i=\pgfpointorigin%
  \let\pgf@nfold@next@tang@ii=\pgfpointorigin%
  \def\pgf@nfold@next@angle@i{0}%
  \def\pgf@nfold@next@angle@ii{0}%
  \let\pgf@nfold@next@first\pgfpointorigin%
  % the following only need to be set if type=c
  \let\pgf@nfold@next@supporta\pgfpointorigin%
  \let\pgf@nfold@next@supportb\pgfpointorigin%
}

\def\pgf@nfold@clampangle{
  % The computed angles are values between 0 and 360, so their difference is between 360 and -360;
  % we want the difference to be between -180 and 180
  \ifdim\pgfmathresult pt<-180pt\relax
    \pgfutil@tempdima=\pgfmathresult pt
    \advance\pgfutil@tempdima by 360pt
    \edef\pgfmathresult{\pgf@sys@tonumber\pgfutil@tempdima}
  \else\ifdim\pgfmathresult pt>180pt\relax
    \pgfutil@tempdima=\pgfmathresult pt
    \advance\pgfutil@tempdima by -360pt
    \edef\pgfmathresult{\pgf@sys@tonumber\pgfutil@tempdima}
  \fi\fi
}



\def\pgf@nfold@parser@moveup{%
  \let\pgf@nfold@prev@type\pgf@nfold@cur@type%
  \let\pgf@nfold@prev@visible\pgf@nfold@cur@visible%
  \let\pgf@nfold@prev@tang@i\pgf@nfold@cur@tang@i%
  \let\pgf@nfold@prev@tang@ii\pgf@nfold@cur@tang@ii%
  \let\pgf@nfold@prev@angle@i\pgf@nfold@cur@angle@i%
  \let\pgf@nfold@prev@angle@ii\pgf@nfold@cur@angle@ii%
  \let\pgf@nfold@prev@first\pgf@nfold@cur@first%
  \let\pgf@nfold@prev@supporta\pgf@nfold@cur@supporta%
  \let\pgf@nfold@prev@supportb\pgf@nfold@cur@supportb%
  \let\pgf@nfold@prev@last\pgf@nfold@cur@last%
  \let\pgf@nfold@cur@type\pgf@nfold@next@type%
  \let\pgf@nfold@cur@visible\pgf@nfold@next@visible%
  \let\pgf@nfold@cur@tang@i\pgf@nfold@next@tang@i%
  \let\pgf@nfold@cur@tang@ii\pgf@nfold@next@tang@ii%
  \let\pgf@nfold@cur@angle@i\pgf@nfold@next@angle@i%
  \let\pgf@nfold@cur@angle@ii\pgf@nfold@next@angle@ii
  \let\pgf@nfold@cur@first\pgf@nfold@next@first%
  \let\pgf@nfold@cur@supporta\pgf@nfold@next@supporta%
  \let\pgf@nfold@cur@supportb\pgf@nfold@next@supportb%
  \let\pgf@nfold@cur@last\pgf@nfold@next@last%
}

%
% Iterating over the parsed soft path
% -----------------------------------
%

\newcount\pgf@nfold@index
\def\pgf@nfold@run@loop{%
  \pgf@nfold@index=\pgf@nfold@order%
  \pgf@nfold@run@loop@%
}

\def\pgf@nfold@run@loop@{%
  \pgf@nfold@loop@inner%
  \advance\pgf@nfold@index by -1\relax
  \ifnum\pgf@nfold@index>0\relax%
    \pgf@nfold@run@loop@%
  \fi%
}

\def\pgf@nfold@loop@inner{%
  \pgfmathsetmacro{\pgf@nfold@shift@fraction}%
    {-1+2*(\pgf@nfold@index-1)/(\pgf@nfold@order-1)}%
  % Do not store shiftamount as a length (i.e. with pt) because we want to use it in \pgfqpointscale{}{}
  \pgfmathsetmacro{\pgf@nfold@shiftamount}{\pgf@nfold@hwidth*\pgf@nfold@shift@fraction}
  % Transformations are already baked into the path; without this call, they would be applied twice
  \pgftransformreset%
  \parsedpath%
  \pgfsyssoftpath@flushcurrentpath%
  \pgf@up@action%
}

% Computes both the width of the component lines into \pgf@x and the distance
% from the center to the outermost line centers into \pgf@y
% from the current values of \pgflinewidth and \pgfinnerlinewidth.
\def\pgf@nfold@compute@widths@from@double{
    \pgf@x=\pgflinewidth\relax%
    \pgf@y=\pgf@x\relax%
    \advance\pgf@x-\pgfinnerlinewidth\relax%
    \advance\pgf@y+\pgfinnerlinewidth\relax%
    \pgf@x=.5\pgf@x\relax%
    \pgf@y=.25\pgf@y\relax%
}

\def\pgf@nfold@render@cached@softpath{%
  \pgfscope% must use a scope, otherwise we break the arrow tips
    % Compute the full and constituent part line widths
    \pgf@nfold@compute@widths@from@double%
    \pgfsetlinewidth\pgf@x%
    \edef\pgf@nfold@hwidth{\the\pgf@y}%
    \pgfprocessround{\pgf@nfold@cachedpath}{\pgf@nfold@cachedpath}% remove tokens from the soft path
    \pgf@nfold@parsearrows%
    \pgf@nfold@parsesoftpath{\pgf@nfold@cachedpath}{\parsedpath}%
    \pgf@nfold@run@loop%
  \endpgfscope%
}


%
% user interface and pgf/TikZ keys
% --------------------------------
%

% Outputs a provided soft path in #1 offset by a distance provided in #2.
\def\pgfoffsetpath#1#2{%
  \begingroup
    \pgfmathsetlengthmacro\pgf@nfold@parsed@hwidth{#2}
    % \pgf@nfold@hwidth must always be positive
    \pgf@x=\pgf@nfold@parsed@hwidth\relax
    \ifdim\pgf@x<0pt\relax
      \pgf@x=-\pgf@x
      \def\pgf@nfold@shift@fraction{-1}
    \else
      \def\pgf@nfold@shift@fraction{1}
    \fi
    \edef\pgf@nfold@parsed@hwidth{\the\pgf@x}
    \pgfoffsetpathqfraction{#1}{\pgf@nfold@parsed@hwidth}{\pgf@nfold@shift@fraction}
  \endgroup
}

% Outputs a provided soft path in #1 offset by #3*#2 where #2 is a length (>= 0 pt)
% and #3 is a number between -1.0 and 1.0. This differs from \pgfoffsetpath{#1}{#2*#3} 
% in how the joins between segments are rendered.  In particular, \pgfoffsetpathfraction{#1}{10pt}{0}
% does *not* yield the original path, but a new path in the centre of #1 drawn at line width 20pt.
%
\def\pgfoffsetpathfraction#1#2#3{%
  \begingroup
    \pgfmathsetlengthmacro\pgf@nfold@parsed@hwidth{#2}
    \pgfmathsetmacro\pgf@nfold@shift@fraction{#3}
    \pgfoffsetpathqfraction{#1}{\pgf@nfold@parsed@hwidth}{\pgf@nfold@shift@fraction}
  \endgroup
}

% This has the same output as the #3-th segment of nfold=#4.
\def\pgfoffsetpathindex#1#2#3#4{%
  \begingroup
    \pgfmathsetmacro\pgf@nfold@shift@fraction{-1+2*(#3-1)/(#4-1)}
    \pgfoffsetpathqfraction{#1}{#2}{\pgf@nfold@shift@fraction}
  \endgroup
}

% A quick version that skips processing the input values
\def\pgfoffsetpathqfraction#1#2#3{%
  \begingroup
    \pgf@x=#2\relax
    \edef\pgf@nfold@hwidth{\the\pgf@x}
    \edef\pgf@nfold@shift@fraction{#3}
    \pgf@x=\pgf@nfold@shift@fraction\pgf@x\relax
    \edef\pgf@nfold@shiftamount{\pgf@sys@tonumber\pgf@x}
    \pgfprocessround{#1}{\pgf@nfold@cachedpath}% remove tokens from the soft path
    \pgf@nfold@parsesoftpath{\pgf@nfold@cachedpath}{\parsedpath}%
  % Transformations are already baked into the path; without this call, they would be applied twice
    \pgftransformreset%
    \parsedpath%
  \endgroup
}


\pgfkeys{
  /pgf/nfold/.code={%
    \pgf@nfold@order=#1\relax%
    \ifnum\pgf@nfold@order<1\relax%
      \pgfutil@packageerror{tikz-nfold}{The key /pgf/nfold must take a value of at least 1, got \the\pgf@nfold@order}{}%
    \fi%
    % If nfold > numcached AND intersections is not loaded AND we draw an Implies tip, we get an error.
    % We check the first two conditions now and set the respective flag
    \ifnum\pgf@nfold@order>\pgf@nfold@intersec@numcached\relax
      \ifdefined\pgfintersectionofpaths\else
        \pgf@nfold@intersectionsnotloadedtrue
      \fi
    \fi
  },
  /pgf/nfold/.default=2
}


% use \tikzset for scoping reasons, does not appear to be equivalent to \pgfset{/tikz/...=...}
\tikzset{
  nfold/.code={
    \edef\pgf@tmp{\noexpand\pgfkeys{/pgf/nfold=#1}}
    % patch \tikz@double@setup to set /pgf/nfold=#1 as well
    \expandafter\expandafter\expandafter\def%
      \expandafter\expandafter\expandafter\tikz@double@setup%
      \expandafter\expandafter\expandafter{\expandafter\tikz@double@setup\pgf@tmp}
  },
  nfold/.default=2,
  scaling nfold/.code={%
    \pgfscope% scope to contain \tikz@double@setup
      \tikz@double@setup
      % extract double distance between line centers into \pgf@x
      \pgf@nfold@compute@widths@from@double
      \pgf@y=2\pgf@y
      % store (order-1)*\pgf@x in \pgf@xa
      \c@pgf@counta=#1
      \advance\c@pgf@counta by -1\relax
      \global\pgf@y=\c@pgf@counta\pgf@y
    \endpgfscope
    \tikzset{
      double distance between line centers=\pgf@y,
      nfold=#1
    }
  },
  scaling nfold/.default=2,
  % This simply defines the key if tikzcd is not loaded, so we don't run into any errors
  commutative diagrams/scaling nfold/.code={
    \pgfscope% scope to contain \tikz@double@setup
      \tikz@double@setup
      % extract double distance between line centers into \pgf@x
      \pgf@nfold@compute@widths@from@double
      % store (order-1)*\pgf@y in \pgf@ya
      \c@pgf@counta=#1
      \advance\c@pgf@counta by -1\relax
      \pgf@ya=\c@pgf@counta\pgf@y
      % compute the label offset, which is (order-2)*\pgf@y + .5*\pgf@x
      \advance\c@pgf@counta by -1\relax
      \pgf@xa=\c@pgf@counta\pgf@y
      \advance\pgf@xa by .5\pgf@x
      % save the results in \pgf@x and \pgf@y
      \global\pgf@x=\pgf@xa
      \global\pgf@y=2\pgf@ya
    \endpgfscope
    \tikzset{
      commutative diagrams/every label/.append style/.expanded={outer sep=\the\pgf@x},
      double distance between line centers=\pgf@y,
      nfold=#1
    }
  },
  commutative diagrams/scaling nfold/.default=2
}

\endinput
